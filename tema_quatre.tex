\section{Variables Aleatòries Contínues}
\subsection{Mesures de probabilitat absolutament contínues. Funció de densitat}
Sigui $(\Omega, \mathcal{A},p)$ un espai de probabilitat, amb probabilitat induïda en $(\real, \mathcal{B},p_{X})$ (quan prenem la variable aleatòria $X$).

\begin{defi}
  Siguin$\mu_{1},\mu_{2}$ mesures sobre un espai de mesura $(X,\mathcal{A})$, diem que $\mu_{1}$ és \textbf{absolutament contínua} respecte $\mu_{2}$ ($\mu_{1}<<\mu_{2}$) si $\forall A\in \mathcal{A}$,
  \[
    \mu_{2}(A)=0 \implies \mu_{1}(A)=0
  \]
\end{defi}

\begin{defi}
  Una variable aleatòria $X$ és \textbf{absolutament contínua} (o contínua per abreujar), si $p_{X}<<\lambda$ ($\lambda$ és la mesura de Lebesgue).
\end{defi}

\begin{obs}
  Les variables aleatòries discretes \underline{no} són absolutament contínues: \\
  Si $p(X=a)=P_{a}>0$ i prenem $B=\setb{a}$, tenim $\lambda(\setb{a})=0$, però $p_{X}(\setb{a})=P_{a}>0$
\end{obs}

El teorema fonamental que ens permet traduir $p_{X}$ (si $X$ és absolutament contínua) a càlculs usant la mesura $\lambda$, és el següent:

\begin{thm}[(Radon-Nikodym)]
  Sigui $(X,\mathcal{A})$ un espai mesurable i $\mu_{1},\mu_{2}$ mesures sobre $(X,\mathcal{A})$ amb $\mu_{1}<<\mu_{2}$. Aleshores, existeix una funció $f_{\mu_{1}}$, $\mu_{2}$-mesurable tal que
  \[
    \forall A \in \mathcal{A}, \, \mu_{1}(A)=\int_{A}f_{\mu_{1}}d\mu_{2}
  \]
  A més, $f_{\mu_{1}}$ és única $\mu_{2}$-gairebé arreu
\end{thm}

\begin{defi}
  La funció $f_{\mu_{1}}$ és la \textbf{funció de densitat} de la mesura $\mu_{1}$ respecte a $\mu_{2}$. \\
  En el nostre context, $X$ és una variable aleatòria absolutament contínua, i
  \[
    \begin{rcases}
      (\real, \mathcal{B},\underbrace{\lambda}_{\mu_{2}})\\
      (\real, \mathcal{B}, \underbrace{p_{X}}_{\mu_{1}})
    \end{rcases}\underset{\mathclap{X abs. cont.}}{\implies} \, p_{X}<<\lambda\overset{R-N}{\implies} \forall B \in \mathcal{B}, \, \boxed{p_{X}(B)=\int_{B}f_{X}d\lambda}
  \]
\end{defi}

\begin{defi}
  La funció $f_{X}$ s'anomena \textbf{funció de densitat de probabilitat} de $X$.
\end{defi}

\begin{obs}
  En la literatura, s'escriu $f_{\mu_{1}}=\dfrac{d\mu_{1}}{d\mu_{2}}$ (Derivada de Radon-Nikodym).
\end{obs}
