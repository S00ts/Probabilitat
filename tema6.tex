\section{Convergència de variables aleatòries}

\subsection{Modes de convergència i equivalències}

Sigui $\{X_n\}_{n\geq1}$ una seqüència de variables aleatòries. Volem donar sentit a la noció de 
convergència de $\{X_n\}_{n\geq1}$ cap a una variable aleatòria $X$ donada.\\
Veurem que això ho podem fer de diverses maneres.

\begin{defi}
  $\{X_n\}_{n\geq1}$ \textbf{convergeix quasi-segurament} cap a $X$, i ho  escriurem $X_n 
  \overset{qs}{\longrightarrow} X$ si
  \[
    p\Big(\underbrace{\setb{\omega \in \Omega : X_n(\omega)\underset{n}{\longrightarrow}X(\omega)}}_{A}\Big) = 1
  \]
  La definició té sentit perquè $A$ és un succés. Vegem-ho:
  \[
    A_n(m) = \setb{\omega \in \Omega : \abs{X_n(\omega) - X(\omega)} < \frac{1}{m}} \text{ és un succés.}
  \]
  \[
    \implies A(m) = \liminf_n A_n(m) = \setb{\omega \in \Omega : \omega \in A_n(m) \quad 
    \forall n \geq n_0(\omega)} \text{ (és un succés)}
  \]
  Finalment,
  \[
    A = \bigcap_{m\geq1}A(m) = \setb{\omega\in\Omega : \forall m \, \exists n_0(\omega) 
    \text{ tal que } \abs{X_n(\omega) - X(\omega)} < \frac{1}{m} \text{ si } n \geq n_0(\omega)} \text{ és un succés.}
  \]
\end{defi}

\begin{defi}
  Direm que $\setb{X_n}_{n\geq1}$ \textbf{convergeix en mitjana d'ordre r} cap a $X$ 
  ($r \geq 1$) i ho escriurem $X_n \overset{r}{\longrightarrow} X$ si
  \[
    \E\left[\abs{X_n - X}^r\right] \underset{n}{\longrightarrow} 0
  \]
\end{defi}

\begin{defi}
  Direm que $\setb{X_n}_{n\geq1}$ \textbf{convergeix en probabilitat} cap a $X$ i ho escriurem 
  $X_n \overset{p}{\longrightarrow} X$ si $\forall \epsilon > 0$
  \[
    p\left(\abs{X_n - X} > \epsilon \right) \underset{n}{\longrightarrow} 0
  \]
\end{defi}

\begin{defi}
  Direm que $\setb{X_n}_{n\geq1}$ \textbf{convergeix en distribució} cap a $X$ i ho escriurem 
  $X_n \overset{d}{\longrightarrow} X$ si
  \[
    F_{X_n}(x) \underset{n}{\longrightarrow} F(x) \text{ en els punts de continuïtat de F(x)}
  \]
\end{defi}

\begin{obs}
  Ens cal que $x$ sigui un punt de continuïtat de $F$ per a que famílies de variables aleatòries 
  convergeixin de manera natural en distribució.
\end{obs}

\newpage

\begin{example}
  Sigui $X$ una variable aleatòria fixada, $X_n = X + \dfrac{1}{n}$. Aleshores volem que 
  $X_n \overset{d}{\longrightarrow} X$. Si això passa, 
  \[
    F_{X_n}(x) = p(X_n \leq x) = p(X \leq x -\frac{1}{n}) \underset{n\to \infty}{\longrightarrow}p(X<x) = F_X(x^-) 
    \quad \text{(límit per l'esquerra)}
  \]
  $\implies$ per tal que $F_X(x)$ sigui $F_X(x^-)$ ens cal suposar que $x$ és un punt de continuïtat de $F_X$.
\end{example}
\-\\\\
El que veurem ara  són implicacions entre els diversos modes de convergència. \\
També veurem que les implicacions contraries mai seran certes.

\subsubsection{Diagrama modes de convergència}

\begin{tikzcd}[arrows=Rightarrow]
X_n \overset{qs}{\longrightarrow} X \arrow[rr, "\text{IV}"] & & X_n \overset{p}{\longrightarrow} X  
\arrow[r, "\text{I}"] & X_n \overset{d}{\longrightarrow} X\\
X_n \overset{r}{\longrightarrow} X \arrow[r, "r>s\geq 1"', "\text{III}"]  & X_n \overset{s}{\longrightarrow} 
X \arrow[ur, "\text{II}"']
\end{tikzcd}

\begin{prop}[(I)]
  $X_n \overset{p}{\to} X \implies X_n \overset{d}{\to} X$, i el recíproc \underline{no} és cert. \\\\
  
  Vegem un contraexemple de que el recíproc no és cert: \\
  
  $X \sim Ber(\dfrac{1}{2})$, $X_n = X$, $Y = 1-X$ (En particular, $Y\sim Ber(\dfrac{1}{2})$). \\
  
  És clar que $X_n \overset{d}{\to} Y$, ja que $X_n \overset{d}{\to} X$ i $X$ i $Y$ tenen la mateixa 
  funció de distribució.
  
  Per altra banda, $\abs{X_n - Y} = 1$ (ja que si una val $1$, l'altra val $0$)
  $$\implies p(\abs{X_n-Y} > \varepsilon) = 1 \quad \text{ si } \varepsilon \text{ és prou petit!!}$$
\end{prop}

\begin{prop}[(II)]
  $X_n \overset{1}{\to} X \implies X_n \overset{p}{\to} X$, i el recíproc \underline{no} és cert. \\\\
  
  Vegem un contraexemple del recíproc: \\
  
  \[
    X_n = \begin{cases}
      n^3 &\text{ amb probabilitat } \frac{1}{n^2} \\
      0   &\text{ amb probabilitat } 1 - \frac{1}{n^2}
    \end{cases}
  \]
  \-\\\\
  
  
  El candidat a límit és $X=0$:
  
  \[ 
    \forall \varepsilon > 0, \, p(\abs{X_n - X} > \varepsilon) = p(\abs{X_n} > \varepsilon) = 
    \frac{1}{n^2} \underset{n\to \infty}{\longrightarrow} 0
  \]
  Amb això, $X_n \overset{p}{\to} 0 = X$. Ara bé,
  \[
    \E\left[\abs{X_n - X}\right] = \E\left[\abs{X_n}\right] = n^3 \cdot \frac{1}{n^2} = 
    n \underset{n \to \infty}{\longrightarrow} \infty \implies X_n \not\overset{1}{\to} 0
  \]
\end{prop}

\begin{prop}[(III)]
  Si $r\geq s \geq 1$ i $ X_n \overset{r}{\to} X \implies X_n \overset{s}{\to} X$.\\\\
  
  El recíproc \underline{no} és cert:
  
  \[
    X_n = \begin{cases}
      n &\text{ amb probabilitat } n^{\frac{-(r+s)}{2}} \\
      0 &\text{ amb probabilitat } 1 - n^{\frac{-(r+s)}{2}}
    \end{cases} 
  \]
  Prendrem com a candidat a límit $X=0$.\\
  
  $\E\left[\abs{X_n}^s \right] = n^s \cdot n^{\frac{-(r+s)}{2}} + 0\cdot (1 - n^{\frac{-(r+s)}{2}}) = 
  n^{\frac{s-r}{2}} \underset{n \to \infty}{\longrightarrow} 0 \text{ (perquè } s \leq r)$
  
  $\E\left[\abs{X_n}^r \right] = n^r \cdot n^{\frac{-(r+s)}{2}} + 0\cdot (1 - n^{\frac{-(r+s)}{2}}) = 
  n^{\frac{r-s}{2}} \underset{n \to \infty}{\longrightarrow} +\infty \not= 0 \text{ !!}$
\end{prop}

\begin{lema}
  \-\\
  Sigui $\varepsilon > 0$. Definim $$A_n(\varepsilon) = \setb{\omega \in \Omega : \abs{X_n(\omega) - 
  X(\omega)} > \varepsilon}$$ i $$B_n(\varepsilon) = \bigcup_{m\geq n} A_m(\varepsilon)$$
  
  Aleshores,
  \[
    X_n \overset{q.s.}{\to} X \iff \forall \varepsilon > 0, \, \lim_n p(B_n(\varepsilon)) = 0
  \]
\end{lema}

Com a conseqüència, tenim el següent corol·lari:

\begin{col}
  Amb la notació anterior, 
  \[
    \forall \varepsilon > 0, \sum_{n\geq 1}p(A_n(\varepsilon)) \implies X_n \overset{q.s.}{\to} X
  \]
\end{col}

\newpage

\begin{prop}[(IV)]
  $X_n \overset{q.s.}{\to} X \implies X_n \overset{p}{\to} X$. \\\\
  
  El recíproc \underline{no} és cert:
  
  \[
    X_n = \begin{cases}
      1 &\text{ amb probabilitat } \frac{1}{n}\\
      0 &\text{ amb probabilitat } 1 - \frac{1}{n}
    \end{cases}
  \]
  Candidat a límit: $X=0$
  
  Convergeix en probabilitat: 
  $$\forall \varepsilon > 0, \, p(\abs{X_n - X} > \varepsilon) = p(\abs{X_n}>\varepsilon) 
  \underset{\text{si } \varepsilon < 1}{=} \frac{1}{n} \underset{n \to \infty}{\longrightarrow} 0$$
  
  Vegem ara que no es compleix que $X_n \overset{q.s.}{\to} 0$. Ho veurem calculant $\lim\limits_{n} 
  p(B_n(\varepsilon))$.
  
  Calculem-ho:
  
  Sigui $A_n(\varepsilon) = \setb{\omega \in \Omega : \abs{X_n(\omega) - X(\omega)} > \varepsilon}$
  
    \begin{align*}
        p(B_n(\varepsilon)) &= p\left(\bigcup_{m\geq n}A_m(\varepsilon)\right) = 
        1 - p\left(\bigcap_{m\geq n}\overline{A_m(\varepsilon)}\right) \underset{\substack{A_n(\varepsilon) \\ 
        \text{ indep.}}}{=} 1 - \prod_{m\geq n}{p\left(\overline{A_m(\varepsilon)} \right)} \underset{\text{si } 
        \varepsilon < 1}{=}\\
        &= 1 - \prod_{m\geq n}\left(1-\dfrac{1}{m} \right) = 0
    \end{align*}
    
    Vegem ara que $\lim\limits_{n\to \infty}\displaystyle\prod_{m\geq n}\left(1-\dfrac{1}{m} \right) = 0$
    
    \[
      0 \leq \prod_{m\geq n}\left(1-\dfrac{1}{m} \right) \underset{1-x\leq e^{-x}}{\leq} 
      \exp\left(\sum_{m\geq n}\frac{-1}{m}\right) \underset{\forall n}{=} 0
    \]
    
    Per tant, $\forall n$, $\displaystyle\prod_{m\geq n}\left(1-\dfrac{1}{m} \right) = a_n = 0 
    \implies \lim\limits_{n\to \infty} a_n = 0 \implies p(B_n(\varepsilon)) \underset{n}{\longrightarrow} 1$
\end{prop}

Finalment, veurem una implicació en sentit contrari:

\begin{prop}
  Si $X_n \overset{p}{\to} X$, aleshores existeix una subsuccessió $\setb{n_i}_{i\geq 1}$ per 
  la qual $Z_{n_i}\overset{q.s.}{\longrightarrow} X$.
\end{prop}

\newpage

\subsection{Convergència quasi-segura i la llei forta dels grans nombres}

Comencem amb les propietats bàsiques de la convergència quasi-segura:

\begin{prop}
  Si $X_n \overset{q.s.}{\to} X$ i $Y_n \overset{q.s.}{\to} Y$, aleshores:
  \begin{enumerate}[label=\alph*)]
    \item $X_n + Y_n \overset{q.s.}{\longrightarrow} X+Y$
    \item $X-n \cdot Y_n \overset{q.s.}{\longrightarrow} X \cdot Y$
    \item $c \cdot X_n \overset{q.s.}{\longrightarrow} c \cdot X$ ($c\in \real$)
    \item Si $g$ contínua, $g(X_n) \overset{q.s.}{\longrightarrow} g(X)$
  \end{enumerate}
\end{prop}

El resultat més important en aquest mode de convergència és la llei forta dels grans nombres:

\begin{thm}[(Llei forta dels grans nombres)]
  Sigui $\setb{X_n}_{n\geq 1}$ una successió de variables aleatòries independents i 
  idènticament distribuïdes (i.i.d.) amb X.
  
  A més, $\E[\abs{X}]<+\infty, \, \E[X] = \mu$. Aleshores, si $S_n = X_1 + \ldots + X_n$:
  
  \[
    \frac{S_n}{n} \overset{q.s.}{\longrightarrow}\mu 
  \]
\end{thm}

\begin{obs}
  Sota les mateixes condicions, $\dfrac{S_n}{n}\overset{p}{\to}\E[X] \equiv$ \textit{Llei feble dels grans nombres}
\end{obs}

\begin{obs}
  Sota les mateixes condicions (i també $\V ar[X]< +\infty$), podem afirmar convergència en 
  mitjana d'ordre 2 (o mitjana quadràtica): $\frac{S_n}{n} \overset{2}{\to}\mu$
\end{obs}

\subsection{Convergència en distribució i el teorema central del límit}

Malgrat que la convergència en distribució és el mode més feble de convergència, moltes de les
propietats que hem vist en la convergència quasi-segura no s'hereten en aquest mode de convergència.

\begin{example}
  No és cert que $X_n \overset{d}{\to} X$, $Y_n \overset{d}{\to} Y \implies$ $X_n + Y_n \overset{d}{\to} X+Y$
  
  Per exemple, si prenem $X_n, \, Y_n \sim X$, i la variable aleatòria $X$ es defineix com:
  
  \[
    \begin{cases}
      p(X=0) = \frac{1}{2} \\
      p(X=1) = \frac{1}{2}
    \end{cases}
  \]
  Aleshores,
  
  \[
    \begin{cases}
      p(X_n + Y_n = 0) = \frac{1}{4} \\
      p(X_n + Y_n = 1) = \frac{1}{2} \\
      p(X_n + Y_n = 2) = \frac{1}{4}
    \end{cases}
  \]
  
  i tenim
  
  \[
  \begin{rcases}
      X_n \overset{d}{\to} X \\
      Y_n \overset{d}{\to} Y
    \end{rcases}
    X_n + Y_n \not\overset{d}{\to} 2X
  \]
  Ja que $p(2X = 0) = p(2X=2) = \frac{1}{2}$
  
  La convergència en distribució no distingeix les dues $X$'s en $X + X = 2X$
\end{example}

Malgrat això, quan un dels límits és determinista, aleshores sí podem afirmar convergència en 
distribució:

\begin{prop}
  Siguin $\setb{X_n}_{n\geq1}$, $\setb{Y_n}_{n\geq1}$ seqüències de variables aleatòries tals que 
  $X_n \overset{d}{\to} X$ ($X$ v.a.) i $Y_N \overset{p}{\to} \alpha$ ($\alpha \in \real$), aleshores
  \[
    X_n + Y_n \overset{d}{\to} X + \alpha
  \]
\end{prop}

El que estem veient és que la convergència en distribució no funciona quan combinem seqüències diferents de
variables aleatòries. El que sí veurem és que si $S_n \overset{d}{\to} X$ i $g:\real \to \real$ és contínua
$\implies g(X_n) \overset{d}{\to} g(X)$. Aquest resultat serà conseqüència del següent teorema:

\begin{thm}[de representació de Skorokhod]
  Sigui $\setb{X_n}_{n\geq1}$ una seqüència de variables aleatòries que compleix $X_n \overset{d}{\to} X$.
  Aleshores existeix un espai de probabilitat $(\Omega', \mathcal{A}', p')$ i una seqüència 
  de variables aleatòries $\setb{Y_n}_{n\geq1}$ i $Y$ definides sobre $(\Omega', \mathcal{A}', p')$ tal que:
  
  \begin{enumerate}
      \item $F_{X_n}(x) = F_{Y_n}(x) \, \forall n$, $F_X(x) = F_Y(x) \, \forall x \in \real$
      \item $Y_n \overset{q.s.}{\to} Y$
  \end{enumerate}
\end{thm}

\begin{col}
  Siguin $\setb{X_n}_{n\geq1}$, $X$ v.a. tals que $X_n \overset{d}{\to} X$ i $g: \real \to \real$ és
  contínua, aleshores:
  \[
    g(X_n) \overset{d}{\to} g(X)
  \]
\end{col}

Per a demostrar resultats de convergència en distribució, és molt útil l'ús de funcions característiques.
En particular, tenim el següent teorema que ens permet estudiar el límit puntual de límits de funcions característiques.

\begin{thm}[(Continuïtat de Lévy)]
  Siguin $\setb{X_n}_{n\geq1}$ successió de variables aleatòries, $X$ una v.a., $\Phi_n(t)$ 
  funció característica de $X_n$ i $\Phi$ funció característica de X.
  
  \begin{enumerate}
      \item Si $X_n \overset{d}{\to} X$, aleshores $\Phi_n(t)$ convergeix puntualment a $\Phi(t)$
      \item Si $\Phi_n(t)$ convergeix puntualment cap a $\psi(t)$ i $\psi(t)$ és contínua en $t=0$, 
      aleshores $\exists Y$ variable aleatòria tal que $\Phi_Y(t) = \psi(t)$ i $X_n \overset{d}{\to} Y$
  \end{enumerate}
  
\end{thm}

Aquest resultat ens permet treballar amb funcions característiques (enlloc de funcions de distribució) per
demostrar convergència en distribució.

La primera aplicació és la versió de la llei dels grans nombres amb convergència en distribució:

\begin{thm}[(Versió dèbil de la llei forta dels grans nombres)]
  Sigui $X$ una variable aleatòria tal que $\E[X] = \mu < +\infty$
  
  Sigui $\setb{X_i}_{i\geq1}$ successió de variables aleatòries independents amb $X_i \sim X$.
  
  Sigui $S_n = \sum\limits_{i=1}^n X_i$.
  
  Aleshores,
  \[
    \frac{S_n}{n} \overset{d}{\to} \mu
  \]
\end{thm}

\begin{thm}[central del límit]
  Sigui $X$ una variable aleatòria amb $\E[X] = \mu < +\infty$, $\V ar[X] = \sigma^2 < +\infty$.
  
  Sigui $\setb{X_n}_{n\geq1}$ successió de variables aleatòries independents amb $X_i \sim X$.
  
  Aleshores, 
  
  \[
    \frac{S_n -n\mu}{\sqrt{n\sigma^2}} \overset{d}{\to} N(0,1)
  \]
\end{thm}

\begin{example}
  Si fem $X_n = Bin(n,p)$, aleshores
  
  $$X_n = Y_1 + \ldots + Y_n, \, \setb{Y_i}_{i=1}^n \text{ són independents i } Y_i \sim Ber(p)$$
  
  $$\implies \text{El teorema central del límit diu que } \frac{X_n - np}{\sqrt{np(1-p)}} \overset{d}{\to} N(0,1)$$
\end{example}

\subsection{Teorema de Chernoff i aplicacions}

El que veurem ara és la velocitat de convergència dels termes $\frac{Sn}{n}$ cap a $\mu$ sota condicions 
addicionals de les variables aleatòries $X_i$ (Recordem que hem vist que $S_n = \sum\limits_{i=1}^n X_i$, $X_i\sim X$, $X_i$ 
independents i $\E[X] = \mu$)

En el cas general, si usem Chebyshev, tenim que:

\[
  p(\abs{S_n - n\mu} \geq \varepsilon) < \frac{n\sigma^2}{\varepsilon^2} \qquad (\V ar[X] = \sigma^2)
\]

\[
  p\big(\underbrace{\abs{S_n/n - \mu} \geq \varepsilon}_{B_n(\varepsilon)}\big) < \frac{\sigma^2}{n\varepsilon^2}
\]

$\implies$ Si ara estudiem $\sum\limits_{n\geq1}p(\abs{S_n/n - \mu} \geq \varepsilon) < 
\dfrac{\sigma^2}{\varepsilon^2}\cdot \sum\limits_{n\geq1} \frac{1}{n}$ No ens diu res perquè divergeix.

\newpage

Per tant, Chebyshev \underline{no} és suficientment fort per assegurar que $\sum\limits_{n\geq1}p(B_n(\varepsilon)) < +\infty$

(per poder assegurar convergència q.s.) \\

El que veurem ara és que podem millorar Chebyshev si demanem condicions addicionals sobre els $X_i$.

\begin{thm}[(Fites de Chernoff)]
  Siguin $\setb{X_i}_{i\geq1}$ variables aleatòries independents, $X_i \sim Ber(p_i)$. \\
  
  Sigui $S_n = X_1 + \ldots + X_n$ amb $\E[S_n] = \mu \quad (\mu = p_1 + \ldots + p_n)$ \\
  
  Aleshores, $\forall \delta>0$:
  \begin{enumerate}
      \item Fita cua superior: $p\big(S_n \geq (1+\delta)\mu\big) \leq \exp\left(-\dfrac{\delta^2}{2+\delta}\mu\right)$
      \item Fita cua inferior: $\hspace{0.18cm}p\big(S_n\leq(1-\delta)\mu \big) \leq \exp\left(-\dfrac{\delta^2}{2}\mu\right)$
  \end{enumerate}
\end{thm}

Vegem l'anàleg de la desigualtat de Chebyshev en aquest context:

\begin{col}
  Amb la notació del teorema anterior:
  \[
    p\left(\abs{S_n - \mu}\geq\delta\mu\right) \leq 2\cdot \exp\left(\dfrac{-\delta^2}{3}\mu \right)
  \]
\end{col}

\begin{defi}
  Sigui $A\subseteq \n$ (possiblement infinit). 
  
  La \textbf{funció de representació} de $A$ és $r_A:\n \to \n$ definida per:
  
  \[
    r_A(n) = \abs{\setb{n=a+b : (a,b) \in A^2}}
  \]
\end{defi}

És difícil construir un conjunt $A$ infinit pel qual:

\begin{itemize}
    \item $r_A(n) > 0 \, \forall n \geq n_0$.
    \item $r_A(n)$ sigui fitada.
\end{itemize}

Vegem el següent resultat de Pál Erd{\H o}s, que és el millor que se sap:

\begin{thm}[(Erd{\H o}s)]
  $\exists A\subseteq \n$, $\abs{A} = +\infty$, tal que $\exists n_0, c_1, c_2 > 0$ que compleixen:
  \[
    c_1\log n \leq r_A(n) \leq c_2\log n \qquad \forall n\geq n_0
  \]
\end{thm}
