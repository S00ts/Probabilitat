\section{Variables Aleatòries Discretes}

\subsection{Definicions i conceptes relacionats. Funció generadora de probabilitat}

Sigui $(\Omega, \mathcal{A}, p)$ un espai de probabilitat i $X$ una variable aleatòria.

\begin{defi}
  $X$ és una \textbf{variable aleatòria discreta} si $Im(X)$ és numerable. \\
  Si $X$ és una variable aleatòria discreta, $Im(X) = \setb{x_{i}}_{i \geq 1}$. \\ 
  $X$ ve completament determinada pels valors $p(X=x_{i}) = p_{i}$.
\end{defi}

\begin{defi}
  Anomenem \textbf{funció de distribució} de X a:
  \[
    F_{X}(x)=p(X \leq x) = \sum_{x_{i}\leq x}p_{i}
  \]
\end{defi}

\begin{defi}
  Sigui $A \in \mathcal{B}$, la \textbf{mesura de probabilitat induïda sobre} $\real$ és:
  \[
    p_{X}(A) = \sum_{\substack{x_{i} \in A \\ x_{i}\in Im(X)}}p_{i}
  \]
  En particular, si $Im(X)\cap A = \varnothing$ aleshores $p_{X}(A) = 0 \implies$ obtenim una probabilitat puntual (hi ha punts de $\real$ amb probabilitat $> 0$, a diferència de la mesura de Lebesgue).
\end{defi}

\begin{defi}
  Definim l'\textbf{esperança matemàtica} com: 
  \[
    \E [X] = \int_{\Omega}Xdp \underset{\text{def. de }\int_{\Omega}}{=\joinrel=\joinrel=\joinrel=\joinrel=} \sum_{i\geq 1}x_{i}\cdot p(X=x_{i}) = \sum_{i\geq 1}x_{i}\cdot p_{i}
  \]
  Més en general, si $g(x)$ és una funció mesurable, 
  \[
    \E [g(X)] = \sum_{i\geq 1}g(x_{i})\cdot p(X=x_{i})
  \]
  En particular, 
  \[
    \E[X^{k}]=\sum_{i\geq 1}x_{i}^{k}\cdot p_{i}
  \]
  \[
    \V ar[X] = \E[X^{2}]-\E[X]^{2}=\sum_{i\geq 1}x_{i}^{2}\cdot p_{i} - \Big(\sum_{i\geq 1}x_{i}\cdot p_{i} \Big)^{2}
  \]
\end{defi}

\begin{defi}
  Prenem $X$, $Y$ variables aleatòries discretes amb $Im(X), \, Im(Y)$ numerables. \\
  El \textbf{vector de variables aleatòries} $(X,Y)$ ve completament caracteritzat per: 
  \[
    p_{(X,Y)}(x_{i},y_{j}) = p(X=x_{i}, Y=y_{j}) \qquad (x_{i}\in Im(X); y_{i}\in Im(Y))
  \]
\end{defi}

\newpage
Aleshores, tenim les següents propietats: \\\\
(i) $X$ i $Y$ són independents $\iff \forall x_{i}\in Im(X), \, \forall y_{j}\in Im(Y), \, p_{(X, Y)}(x_{i},y_{j}) = p_{X}(x_{i})\cdot p_{Y}(y_{j})$ \\\\
(ii) Si $X, \, Y$ són independents, $\E[X\cdot Y] = \E[X]\cdot \E[Y]$
