\section{Convergència de variables aleatòries}

\subsection{Modes de convergència i equivalències}

Sigui $\{X_n\}_{n\geq1}$ una seqüència de variables aleatòries. Volem donar sentit a la noció de convergència de $\{X_n\}_{n\geq1}$ cap a una variable aleatòria $X$ donada.\\
Veurem que això ho podem fer de diverses maneres.

\begin{defi}
  $\{X_n\}_{n\geq1}$ \textbf{convergeix quasi-segurament} cap a $X$, i ho  escriurem $X_n \overset{qs}{\longrightarrow} X$ si
  \[
    p\Big(\underbrace{\setb{\omega \in \Omega : X_n(\omega)\underset{n}{\longrightarrow}X(\omega)}}_{A}\Big) = 1
  \]
  La definició té sentit perquè $A$ és un succés. Vegem-ho:
  \[
    A_n(m) = \setb{\omega \in \Omega : \abs{X_n(\omega) - X(\omega)} < \frac{1}{m}} \text{ és un succés.}
  \]
  \[
    \implies A(m) = \liminf_n A_n(m) = \setb{\omega \in \Omega : \omega \in A_n(m) \quad \forall n \geq n_0(\omega)} \text{ (és un succés)}
  \]
  Finalment,
  \[
    A = \bigcap_{m\geq1}A(m) = \setb{\omega\in\Omega : \forall m \, \exists n_0(\omega) \text{ tal que } \abs{X_n(\omega) - X(\omega)} < \frac{1}{m} \text{ si } n \geq n_0(\omega)} \text{ és un succés.}
  \]
\end{defi}

\begin{defi}
  Direm que $\setb{X_n}_{n\geq1}$ \textbf{convergeix en mitjana d'ordre r} cap a $X$ ($r \geq 1$) i ho escriurem $X_n \overset{r}{\longrightarrow} X$ si
  \[
    \E\left[\abs{X_n - X}^r\right] \underset{n}{\longrightarrow} 0
  \]
\end{defi}

\begin{defi}
  Direm que $\setb{X_n}_{n\geq1}$ \textbf{convergeix en probabilitat} cap a $X$ i ho escriurem $X_n \overset{p}{\longrightarrow} X$ si $\forall \epsilon > 0$
  \[
    p\left(\abs{X_n - X} > \epsilon \right) \underset{n}{\longrightarrow} 0
  \]
\end{defi}

\begin{defi}
  Direm que $\setb{X_n}_{n\geq1}$ \textbf{convergeix en distribució} cap a $X$ i ho escriurem $X_n \overset{d}{\longrightarrow} X$ si
  \[
    F_{X_n}(x) \underset{n}{\longrightarrow} F(x) \text{ en els punts de continuïtat de F(x)}
  \]
\end{defi}

\begin{obs}
  Ens cal que $x$ sigui un punt de continuïtat de $F$ per a que famílies de variables aleatòries convergeixin de manera natural en distribució.
\end{obs}

\newpage

\begin{example}
  Sigui $X$ una variable aleatòria fixada, $X_n = X + \dfrac{1}{n}$. Aleshores volem que $X_n \overset{d}{\longrightarrow} X$. Si això passa, 
  \[
    F_{X_n}(x) = p(X_n \leq x) = p(X \leq x -\frac{1}{n}) \underset{n\to \infty}{\longrightarrow}p(X<x) = F_X(x^-) \quad \text{(límit per l'esquerra)}
  \]
  $\implies$ per tal que $F_X(x)$ sigui $F_X(x^-)$ ens cal suposar que $x$ és un punt de continuïtat de $F_X$.
\end{example}
\-\\\\
El que veurem ara  són implicacions entre els diversos modes de convergència. \\
També veurem que les implicacions contraries mai seran certes.

\subsubsection{Diagrama modes de convergència}

\begin{tikzcd}[arrows=Rightarrow]
X_n \overset{qs}{\longrightarrow} X \arrow[rr, "\text{IV}"] & & X_n \overset{p}{\longrightarrow} X  \arrow[r, "\text{I}"] & X_n \overset{d}{\longrightarrow} X\\
X_n \overset{r}{\longrightarrow} X \arrow[r, "r>s\geq 1"', "\text{III}"]  & X_n \overset{s}{\longrightarrow} X \arrow[ur, "\text{II}"']
\end{tikzcd}
