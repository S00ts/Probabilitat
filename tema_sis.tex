\section{Convergència de variables aleatòries}

\subsection{Modes de convergència i equivalències}

Sigui $\{X_n\}_{n\geq1}$ una seqüència de variables aleatòries. Volem donar sentit a la noció de convergència de $\{X_n\}_{n\geq1}$ cap a una variable aleatòria $X$ donada.\\
Veurem que això ho podem fer de diverses maneres.

\begin{defi}
  $\{X_n\}_{n\geq1}$ \textbf{convergeix quasi-segurament} cap a $X$, i ho  escriurem $X_n \overset{qs}{\longrightarrow} X$ si
  \[
    p\Big(\underbrace{\setb{\omega \in \Omega : X_n(\omega)\underset{n}{\longrightarrow}X(\omega)}}_{A}\Big) = 1
  \]
  La definició té sentit perquè $A$ és un succés. Vegem-ho:
  \[
    A_n(m) = \setb{\omega \in \Omega : \abs{X_n(\omega) - X(\omega)} < \frac{1}{m}} \text{ és un succés.}
  \]
  \[
    \implies A(m) = \liminf_n A_n(m) = \setb{\omega \in \Omega : \omega \in A_n(m) \quad \forall n \geq n_0(\omega)} \text{ (és un succés)}
  \]
  Finalment,
  \[
    A = \bigcap_{m\geq1}A(m) = \setb{\omega\in\Omega : \forall m \, \exists n_0(\omega) \text{ tal que } \abs{X_n(\omega) - X(\omega)} < \frac{1}{m} \text{ si } n \geq n_0(\omega)} \text{ és un succés.}
  \]
\end{defi}

\begin{defi}
  Direm que $\setb{X_n}_{n\geq1}$ \textbf{convergeix en mitjana d'ordre r} cap a $X$ ($r \geq 1$) i ho escriurem $X_n \overset{r}{\longrightarrow} X$ si
  \[
    \E\left[\abs{X_n - X}^r\right] \underset{n}{\longrightarrow} 0
  \]
\end{defi}

\begin{defi}
  Direm que $\setb{X_n}_{n\geq1}$ \textbf{convergeix en probabilitat} cap a $X$ i ho escriurem $X_n \overset{p}{\longrightarrow} X$ si $\forall \epsilon > 0$
  \[
    p\left(\abs{X_n - X} > \epsilon \right) \underset{n}{\longrightarrow} 0
  \]
\end{defi}

\begin{defi}
  Direm que $\setb{X_n}_{n\geq1}$ \textbf{convergeix en distribució} cap a $X$ i ho escriurem $X_n \overset{d}{\longrightarrow} X$ si
  \[
    F_{X_n}(x) \underset{n}{\longrightarrow} F(x) \text{ en els punts de continuïtat de F(x)}
  \]
\end{defi}

\begin{obs}
  Ens cal que $x$ sigui un punt de continuïtat de $F$ per a que famílies de variables aleatòries convergeixin de manera natural en distribució.
\end{obs}

\newpage

\begin{example}
  Sigui $X$ una variable aleatòria fixada, $X_n = X + \dfrac{1}{n}$. Aleshores volem que $X_n \overset{d}{\longrightarrow} X$. Si això passa, 
  \[
    F_{X_n}(x) = p(X_n \leq x) = p(X \leq x -\frac{1}{n}) \underset{n\to \infty}{\longrightarrow}p(X<x) = F_X(x^-) \quad \text{(límit per l'esquerra)}
  \]
  $\implies$ per tal que $F_X(x)$ sigui $F_X(x^-)$ ens cal suposar que $x$ és un punt de continuïtat de $F_X$.
\end{example}
\-\\\\
El que veurem ara  són implicacions entre els diversos modes de convergència. \\
També veurem que les implicacions contraries mai seran certes.

\subsubsection{Diagrama modes de convergència}

\begin{tikzcd}[arrows=Rightarrow]
X_n \overset{qs}{\longrightarrow} X \arrow[rr, "\text{IV}"] & & X_n \overset{p}{\longrightarrow} X  \arrow[r, "\text{I}"] & X_n \overset{d}{\longrightarrow} X\\
X_n \overset{r}{\longrightarrow} X \arrow[r, "r>s\geq 1"', "\text{III}"]  & X_n \overset{s}{\longrightarrow} X \arrow[ur, "\text{II}"']
\end{tikzcd}

\begin{prop}[(I)]
  $X_n \overset{p}{\to} X \implies X_n \overset{d}{\to} X$, i el recíproc \underline{no} és cert. \\\\
  
  Vegem un contraexemple de que el recíproc no és cert: \\
  
  $X \sim Ber(\dfrac{1}{2})$, $X_n = X$, $Y = 1-X$ (En particular, $Y\sim Ber(\dfrac{1}{2})$). \\
  
  És clar que $X_n \overset{d}{\to} Y$, ja que $X_n \overset{d}{\to} X$ i $X$ i $Y$ tenen la mateixa funció de distribució.
  
  Per altra banda, $\abs{X_n - Y} = 1$ (ja que si una val $1$, l'altra val $0$)
  $$\implies p(\abs{X_n-Y} > \varepsilon) = 1 \quad \text{ si } \varepsilon \text{ és prou petit!!}$$
\end{prop}

\begin{prop}[(II)]
  $X_n \overset{1}{\to} X \implies X_n \overset{p}{\to} X$, i el recíproc \underline{no} és cert. \\\\
  
  Vegem un contraexemple del recíproc: \\
  
  \[
    X_n = \begin{cases}
      n^3 &\text{ amb probabilitat } \frac{1}{n^2} \\
      0   &\text{ amb probabilitat } 1 - \frac{1}{n^2}
    \end{cases}
  \]
  \-\\\\
  
  
  El candidat a límit és $X=0$:
  
  \[ 
    \forall \varepsilon > 0, \, p(\abs{X_n - X} > \varepsilon) = p(\abs{X_n} > \varepsilon) = \frac{1}{n^2} \underset{n\to \infty}{\longrightarrow} 0
  \]
  Amb això, $X_n \overset{p}{\to} 0 = X$. Ara bé,
  \[
    \E\left[\abs{X_n - X}\right] = \E\left[\abs{X_n}\right] = n^3 \cdot \frac{1}{n^2} = n \underset{n \to \infty}{\longrightarrow} \infty \implies X_n \not\overset{1}{\to} 0
  \]
\end{prop}

\begin{prop}[(III)]
  Si $r\geq s \geq 1$ i $ X_n \overset{r}{\to} X \implies X_n \overset{s}{\to} X$.\\\\
  
  El recíproc \underline{no} és cert:
  
  \[
    X_n = \begin{cases}
      n &\text{ amb probabilitat } n^{\frac{-(r+s)}{2}} \\
      0 &\text{ amb probabilitat } 1 - n^{\frac{-(r+s)}{2}}
    \end{cases} 
  \]
  Prendrem com a candidat a límit $X=0$.\\
  
  $\E\left[\abs{X_n}^s \right] = n^s \cdot n^{\frac{-(r+s)}{2}} + 0\cdot (1 - n^{\frac{-(r+s)}{2}}) = n^{\frac{s-r}{2}} \underset{n \to \infty}{\longrightarrow} 0 \text{ (perquè } s \leq r)$
  
  $\E\left[\abs{X_n}^r \right] = n^r \cdot n^{\frac{-(r+s)}{2}} + 0\cdot (1 - n^{\frac{-(r+s)}{2}}) = n^{\frac{r-s}{2}} \underset{n \to \infty}{\longrightarrow} +\infty \not= 0 \text{ !!}$
\end{prop}

\begin{lema}
  \-\\
  Sigui $\varepsilon > 0$. Definim $$A_n(\varepsilon) = \setb{\omega \in \Omega : \abs{X_n(\omega) - X(\omega)} > \varepsilon}$$ i $$B_n(\varepsilon) = \bigcup_{m\geq n} A_m(\varepsilon)$$
  
  Aleshores,
  \[
    X_n \overset{q.s.}{\to} X \iff \forall \varepsilon > 0, \, \lim_n p(B_n(\varepsilon)) = 0
  \]
\end{lema}

Com a conseqüència, tenim el següent corol·lari:

\begin{col}
  Amb la notació anterior, 
  \[
    \forall \varepsilon > 0, \sum_{n\geq 1}p(A_n(\varepsilon)) \implies X_n \overset{q.s.}{\to} X
  \]
\end{col}

\newpage

\begin{prop}[(IV)]
  $X_n \overset{q.s.}{\to} X \implies X_n \overset{p}{\to} X$. \\\\
  
  El recíproc \underline{no} és cert:
  
  \[
    X_n = \begin{cases}
      1 &\text{ amb probabilitat } \frac{1}{n}\\
      0 &\text{ amb probabilitat } 1 - \frac{1}{n}
    \end{cases}
  \]
  Candidat a límit: $X=0$
  
  Convergeix en probabilitat: 
  $$\forall \varepsilon > 0, \, p(\abs{X_n - X} > \varepsilon) = p(\abs{X_n}>\varepsilon) \underset{\text{si } \varepsilon < 1}{=} \frac{1}{n} \underset{n \to \infty}{\longrightarrow} 0$$
  
  Vegem ara que no es compleix que $X_n \overset{q.s.}{\to} 0$. Ho veurem calculant $\lim\limits_{n} p(B_n(\varepsilon))$.
  
  Calculem-ho:
  
  \[
    p(B_n(\varepsilon)) = 
  \]
\end{prop}
