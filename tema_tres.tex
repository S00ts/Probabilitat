\section{Variables Aleatòries Discretes}

\subsection{Definicions i conceptes relacionats. Funció generadora de probabilitat}

Sigui $(\Omega, \mathcal{A}, p)$ un espai de probabilitat i $X$ una variable aleatòria.

\begin{defi}
  $X$ és una \textbf{variable aleatòria discreta} si $Im(X)$ és numerable. \\
  Si $X$ és una variable aleatòria discreta, $Im(X) = \setb{x_{i}}_{i \geq 1}$. \\ 
  $X$ ve completament determinada pels valors $p(X=x_{i}) = p_{i}$.
\end{defi}

\begin{defi}
  Anomenem \textbf{funció de distribució} de X a:
  \[
    F_{X}(x)=p(X \leq x) = \sum_{x_{i}\leq x}p_{i}
  \]
\end{defi}

\begin{defi}
  Sigui $A \in \mathcal{B}$, la \textbf{mesura de probabilitat induïda sobre} $\real$ és:
  \[
    p_{X}(A) = \sum_{\substack{x_{i} \in A \\ x_{i}\in Im(X)}}p_{i}
  \]
  En particular, si $Im(X)\cap A = \varnothing$ aleshores $p_{X}(A) = 0 \implies$ obtenim una probabilitat puntual (hi ha punts de $\real$ amb probabilitat $> 0$, a diferència de la mesura de Lebesgue).
\end{defi}

\begin{defi}
  Definim l'\textbf{esperança matemàtica} com: 
  \[
    \E [X] = \int_{\Omega}Xdp \underset{\text{def. de }\int_{\Omega}}{=\joinrel=\joinrel=\joinrel=\joinrel=} \sum_{i\geq 1}x_{i}\cdot p(X=x_{i}) = \sum_{i\geq 1}x_{i}\cdot p_{i}
  \]
  Més en general, si $g(x)$ és una funció mesurable, 
  \[
    \E [g(X)] = \sum_{i\geq 1}g(x_{i})\cdot p(X=x_{i})
  \]
  En particular, 
  \[
    \E[X^{k}]=\sum_{i\geq 1}x_{i}^{k}\cdot p_{i}
  \]
  \[
    \V ar[X] = \E[X^{2}]-\E[X]^{2}=\sum_{i\geq 1}x_{i}^{2}\cdot p_{i} - \Big(\sum_{i\geq 1}x_{i}\cdot p_{i} \Big)^{2}
  \]
\end{defi}

\begin{defi}
  Prenem $X$, $Y$ variables aleatòries discretes amb $Im(X), \, Im(Y)$ numerables. \\
  El \textbf{vector de variables aleatòries} $(X,Y)$ ve completament caracteritzat per: 
  \[
    p_{(X,Y)}(x_{i},y_{j}) = p(X=x_{i}, Y=y_{j}) \qquad (x_{i}\in Im(X); y_{i}\in Im(Y))
  \]
\end{defi}

\newpage
Aleshores, tenim les següents propietats: \\\\
(i) $X$ i $Y$ són independents $\iff \forall x_{i}\in Im(X), \, \forall y_{j}\in Im(Y), \, p_{(X, Y)}(x_{i},y_{j}) = p_{X}(x_{i})\cdot p_{Y}(y_{j})$ \\\\
(ii) Si $X, \, Y$ són independents, $\E[X\cdot Y] = \E[X]\cdot \E[Y]$ \\\\

Sigui $X$ una variable aleatòria discreta amb $Im(X)\subseteq \n_{\geq0}$. En aquesta situació, tenim la següent definició:

\begin{defi}
  La \textbf{funció generadora de probabilitat associada a $X$} és: 
  \[
    G_{X}(z) = \sum_{i\geq 0}p(X=i)\cdot z^{i} = \sum_{i\geq 0}p_{i}\cdot z^{i}
  \]
  
  Una funció generadora de probabilitat és un objecte formal. En particular, $G_{X}(z)=\E[z^{X}]$.
\end{defi}

Si ens mirem les funcions generadores de probabilitat com funcions (en $\cx$), aleshores compleixen el següent:

\begin{prop}
  \begin{enumerate}
      \item []
      \item $G_{X}(0) = p(X=0)$
      \item $G_{X}(1) = 1$. A més, si $\abs{z}\leq 1 \, (z\in \cx) \implies \abs{G_{X}(z)}\leq 1$ \\
      Per tant, $G_{X}(z)$ (com a sèrie de potències) té radi de convergència $\geq 1$.
      \item $\E[X] = \dfrac{d}{dz}G_{X}(z)_{\big|z=1}$
      \item Més en general, $\E[(X)_{k}] = \dfrac{d^{k}}{dz^{k}}G_{X}(z)_{\big|z=1}$ \\
      En particular, $\V ar[X] = G_{X}''(1) + G_{X}'(1) - (G_{X}'(1))^{2}$
      \item Si $z\in \real, \, z \in [0,1]$, aleshores $G_{X}(z)$ és una funció creixent.
  \end{enumerate}
\end{prop}

La propietat fonamental de les funcions generadores de probabilitat és que permeten estudiar fàcilment sumes de variables aleatòries independents.

\begin{prop}
  Siguin $X, Y$ variables aleatòries discretes independents amb imatge en $\n$, i amb funcions generadores de probabilitat $G_{X}(z), \, G_{Y}(z)$. Aleshores: 
  \[
    G_{X+Y}(z) = G_{X}(z)\cdot G_{Y}(z)
  \]
\end{prop}

\begin{col}
  Si $X_{1}, \ldots, X_{N}$ són variables aleatòries discretes independents amb imatge en $\n$, aleshores: 
  \[
    G_{X_{1}+\ldots+X_{N}}(z) = \prod_{i=1}^{N} G_{X_{i}}(z)
  \]
\end{col}
