\documentclass[12pt]{article}
 
\usepackage[margin=1in]{geometry}
\usepackage[pdftex]{hyperref}
\usepackage{amsmath,amsthm,amssymb,graphicx,mathtools,tikz,hyperref,enumerate}
\usepackage{mdframed,cleveref}
\usepackage{bbm}

\newmdenv[leftline=false,topline=false]{topright}
\let\proof\relax
\usepackage[utf8]{inputenc}
\usetikzlibrary{positioning}
\newcommand{\n}{\mathbb{N}}
\newcommand{\z}{\mathbb{Z}}
\newcommand{\q}{\mathbb{Q}}
\newcommand{\cx}{\mathbb{C}}
\newcommand{\real}{\mathbb{R}}
\newcommand{\E}{\mathbb{E}}
\newcommand{\V}{\mathbb{V}}
\newcommand{\bb}[1]{\mathbb{#1}}
\let\k\relax
\newcommand{\k}{\mathbf{k}}
\newcommand{\ita}[1]{\textit{#1}}
\newcommand\inv[1]{#1^{-1}}
\newcommand\setb[1]{\left\{#1\right\}}
\newcommand{\vbrack}[1]{\langle #1\rangle}
\newcommand{\determinant}[1]{\begin{vmatrix}#1\end{vmatrix}}
\newcommand{\abs}[1]{\left\vert #1 \right\vert}
\DeclareMathOperator{\Id}{Id}


\hypersetup{
	colorlinks,
	linkcolor=blue
}
 
 \renewcommand*\contentsname{Continguts}

\newtheoremstyle{break}% name
{}%         Space above, empty = `usual value'
{}%         Space below
{}% Body font
{}%         Indent amount (empty = no indent, \parindent = para indent)
{\bfseries}% Thm head font
{}%        Punctuation after thm head
{\newline}% Space after thm head: \newline = linebreak
{#1 #2 \normalfont #3}%         Thm head spec

\newtheoremstyle{breakthm}% name
{}%         Space above, empty = `usual value'
{}%         Space below
{}% Body font
{}%         Indent amount (empty = no indent, \parindent = para indent)
{\bfseries}% Thm head font
{}%        Punctuation after thm head
{\newline}% Space after thm head: \newline = linebreak
{#1 \normalfont #3 (#2)\addcontentsline{toc}{subsubsection}{#1 #3}}%         Thm head spec
\newtheoremstyle{normal}% name
{}%         Space above, empty = `usual value'
{}%         Space below
{}% Body font
{}%         Indent amount (empty = no indent, \parindent = para indent)
{\bfseries}% Thm head font
{}%        Punctuation after thm head
{5pt plus 1pt minus 1pt}% Space after thm head: \newline = linebreak
{#1 #2 \normalfont #3}%         Thm head spec

\theoremstyle{normal}
\newtheorem{lema}{Lema}[subsection]
\newtheorem{obs}[lema]{Observació}

\theoremstyle{break}
\newtheorem{prop}[lema]{Proposició}
\newtheorem{properties}[lema]{Propietats}
\newtheorem*{proof}{Demostració}
\newtheorem{defi}[lema]{Definició}
\newtheorem{col}[lema]{Corol·lari}
\newtheorem{ej}[lema]{Exercici}
\newtheorem{example}[lema]{Exemple}

\theoremstyle{breakthm}
\newtheorem{thm}[lema]{Teorema}
\newtheorem{defiImp}[lema]{Definició} %Pq surti al índex



\setcounter{section}{1}

 
\begin{document}
\date{}
\setlength{\parindent}{0pt}

\title{Teoria de la Probabilitat}
 
\maketitle

\tableofcontents

\setcounter{section}{0}

\newpage
\section{Espais de Probabilitat}

\subsection{Definició axiomàtica d'espai de probabilitat}

\begin{defi}
  Un \textbf{espai de probabilitat} és un espai de mesura $(\Omega, \mathcal{A}, p)$, tal que $p(\Omega) = 1$.
  \begin{itemize}
      \item $\Omega$ s'anomena \textbf{espai mostral}.
      \item $\mathcal{A}$ se l'anomena \textbf{conjunt d'esdeveniments} o \textbf{successos}.
      \item $p$ se l'anomena \textbf{funció de probabilitat}.
  \end{itemize}
\end{defi}

\begin{obs}
  $\mathcal{A} \subseteq \mathcal{P}(\Omega)$ és una $\sigma$-àlgebra: \\
  $\sigma1)$ $\varnothing \in \mathcal{A}$ \\
  $\sigma2)$ $A \in \mathcal{A} \iff \overline{A} \in \mathcal{A}$ \\
  $\sigma3)$ Si $\setb{A_{n}}_{n\geq1}$ és una seqüència de successos en $\mathcal{A} \implies \bigcup\limits_{n\geq1}A_{n} \in \mathcal{A}$
\end{obs}

\begin{obs}
  Recordem que $p$ és una mesura i, per tant: \\
  $p1)$ $p(\varnothing) = 0$ \\
  $p2)$ $\forall A \in \mathcal{A},\, p(A)\geq0$ \\
  $p3)$ Si $\setb{A_{n}}_{n\geq1}$ és una seqüència de successos en $\mathcal{A}$ disjunts 2 a 2 $(A_{i}\cap A_{j} = \varnothing \, si \, i \not =j)$, aleshores 
  \[
    p\bigg(\bigcup\limits_{n\geq1}A_{n}\bigg) = \sum_{n\geq1}p(A_{n})
  \]
\end{obs}

Vegem les primeres propietats dels espais de probabilitat:

\begin{prop}
  Per un espai de probabilitat $(\Omega, \mathcal{A}, p)$ es compleix que: \\
  (i) $A\in\mathcal{A} \implies p(\overline{A}) = 1-p(A)$ \\
  (ii) Si $A, B \in \mathcal{A},\, A \subseteq B \implies p(A)\leq p(B)$ \\
  (iii) Si $A_{1},\ldots,A_{r} \in \mathcal{A}, \, i \, A_{i}\cap A_{j} \not= \varnothing \, (i\not= j)$, aleshores $p\bigg(\bigcup\limits_{i=1}^{r}A_{i} \bigg) = \sum\limits_{i=1}^{r}p(A_{i})$ \\
  (iv) Si $A,B \in \mathcal{A}, \, A\subseteq B \implies p(B-A) = p(B)-p(A)$\\
  (v) Successions monòtones: 
  \[
    \hspace{-2.5cm}\text{a) Si } A_{1} \subseteq A_{2} \subseteq A_{3} \subseteq \ldots \subseteq A_{i} \in \mathcal{A} \implies p\bigg(\bigcup\limits_{n\geq1}A_{n}\bigg) = \lim_{n\to\infty}p(A_{n})
  \]
  \[
    \hspace{-2.5cm}\text{b) Si } A_{1} \supseteq A_{2} \supseteq A_{3} \supseteq \ldots \supseteq A_{i} \in \mathcal{A} \implies p\bigg(\bigcap\limits_{n\geq1}A_{n}\bigg) = \lim_{n\to\infty}p(A_{n})
  \]
\end{prop}

\newpage

Si ara tenim un espai de probabilitat $(\Omega, \mathcal{A},p)$, i successos $A_{1}, A_{2}, \ldots, A_{i}$ en general \underline{no} disjunts, aleshores \underline{no} és cert que  $p\bigg(\bigcup\limits_{i=1}^{r}A_{r} \bigg) = \sum\limits_{i=1}^{r}p(A_{i})$. En aquest cas, tenim la següent fita:
\begin{lema}[(Fita de la unió)] \-\\
  Siguin $A_{1}, A_{2}, \ldots, A_{r}$ successos en $(\Omega, \mathcal{A},p)$, aleshores $$p\bigg(\bigcup\limits_{i=1}^{r}A_{i} \bigg) \leq \sum\limits_{i=1}^{r}p(A_{i})$$
\end{lema}

\begin{thm}[(Desigualtats de Bonferroni)]
  Siguin $A_{1}, \ldots, A_{r}$ successos en $(\Omega, \mathcal{A}, p)$. 
  Denotem per $I\subseteq \setb{1,\ldots,r} \colon = [r]$, \\
  \[
    A_{I} = \bigcap\limits_{i\in I}A_{i}
  \]
  \[
    S_{k} = \sum_{\substack{I\subseteq[r] \\ \abs{I}=k}}p(A_{I})
  \]
  Aleshores, si: \\
  1) t és parell, $p\bigg(\bigcup\limits_{i=1}^{r} A_{i}\bigg)\geq \sum\limits_{i=1}^{t}(-1)^{i+1}\cdot S_{i}$\\\\
  2) t és senar, $p\bigg(\bigcup\limits_{i=1}^{r} A_{i}\bigg)\leq \sum\limits_{i=1}^{t}(-1)^{i+1}\cdot S_{i}$
\end{thm}

\begin{example}

\end{example}

\newpage

\subsection{Probabilitat condicionada}
\begin{defi}
  Sigui $(\Omega, \mathcal{A}, p)$ un espai de probabilitat , i $B \in \mathcal{A}$ amb $p(B) > 0$. \\
  Per $A \in \mathcal{A}$, la \textbf{probabilitat condicionada} de A amb B és:
  \[
    p(A\mid B) = \frac{p(A\cap B)}{p(B)}
  \]
\end{defi}

\begin{obs}
  $p(A \mid B)$ mesura la probabilitat de que el succés $A$ ocorri sabent que $B$ ha succeït.
\end{obs}

\begin{obs}
  Si prenem
  \[
  \begin{aligned}
      P_{B} \colon \mathcal{A} &\to \real \\
      A &\mapsto P_{B}(A) = P(A\mid B)
  \end{aligned}
  \]
  Aleshores $P_{B}$ és una funció de probabilitat sobre $\Omega, \mathcal{A}$ \\
  De fet, si definim $\mathcal{A}_{B} = \setb{A\cap B\colon A\in\mathcal{A}}$, aleshores $\mathcal{A}_{B}$ és una $\sigma$-àlgebra i $P_{B}$ també defineix una probabilitat sobre $(\Omega, \mathcal{A}_{B})$.
\end{obs}

\begin{prop}
  Siguin $A_{1}, \ldots, A_{r} \in \mathcal{A}$, tals que $p(A_{i}) > 0, \, A_{i}\cap A_{j} = \varnothing \text{ si } i \not= j$ i $\bigcup\limits_{i = 1}^{r}A_{i} = \Omega \\ \quad (\setb{A_{i}}_{i=1}^{r}$ és una partició de $\Omega)$ \\
  
  (i) \underline{Teorema de la probabilitat total}: $\forall A \in \mathcal{A}, p(A)=\sum\limits_{i=1}^{r}p(A \mid A_{i})\cdot p(A_{i})$. \\
  (ii) \underline{Fórmula de Bayes}: si $A \in \mathcal{A}, \, p(A) > 0,$
  \[
    p(A_{j} \mid A) = \frac{p(A\mid A_{j})\cdot p(A_{j})}{\sum\limits_{i=1}^{r}p(A \mid A_{i})\cdot p(A_{i})}
  \]
\end{prop}

\newpage
\section{Variables Aleatòries}
\subsection{Definició de variable aleatòria. Llei d'una v.a.}
Sigui $(\Omega, \mathcal{A}, \beta)$  un espai de probabilitat. Volem estudiar funcions de $\Omega$ amb imatge en $\real$.

\begin{defi}
  Una \textbf{variable aleatòria} és una funció $X\colon \Omega \to \real$ tal que per tot borelià B 
  $\in \mathcal{B}$, $\inv{X}(B) \in \mathcal{A}$. \\
  
  Per tant, una variable aleatòria és una funció mesurable entre els espais de mesura $(\Omega, \mathcal{A}, p)$ i $(\real, \mathcal{B}, \lambda)$.
\end{defi}

\begin{example}
  (1) Les funcions constants són variables aleatòries: \\
    \[
    \begin{aligned}
      X \colon \Omega &\to \real \\
      \omega &\mapsto c
    \end{aligned}
    \qquad \text{Si prenem } B \in \mathcal{B} \text{, } \inv{X}(B) =
    \begin{cases}
			\emptyset \quad \text{si c} \notin B\\
			\Omega \quad \text{si c} \in B
    \end{cases}
    \]
  \\
  
  (2) \textbf{Variables aleatòries indicadores}: 
    \[
    \hspace{-3.5cm}\text{Sigui A}  \in \mathcal{A}\text{, definim }\mathbbm{1}_{A}\colon \Omega \to \real \text{ on } \mathbbm{1}_{A}(\omega) = 
    \begin{cases}
			0 \quad \text{si } \omega \notin A\\
			1 \quad \text{si } \omega \in A
    \end{cases}\\
    \]
    \[
    \hspace{-3.5cm}\text{Aleshores, } B \in \mathcal{B}, \inv{\mathbbm{1}_{A}}(B) = 
    \begin{cases}
			\emptyset \quad \text{si } \setb{0, 1} \nsubseteq B\\
			A \quad \text{si } 1 \in B, \quad 0\notin B\\
			\overline{A} \quad \text{si } 1 \notin B, \quad 0 \in B\\
			\Omega \quad \text{si } \setb{0, 1} \nsubseteq B
    \end{cases}\\
    \]
    \\
    
    (3) Si X i Y són v.a., aleshores $X + Y$, $X\cdot Y$, $\abs{X}$, etc. són v.a. \\
    \- \hspace{0.5cm}En general, si $g\colon \real^{2} \to \real$ és una funció mesurable, aleshores g(X,Y) és una v.a.\\
\end{example}

Estem dient que $\forall B \in \mathcal{B}$, $\setb{\omega \in \Omega \colon X(\omega) \in B}$ és un succés i, per tant, podem calcular $P(\setb{\omega \in \Omega \colon X(\omega) \in B}) \equiv P(X \in B)$.\\

\begin{example}
  $P(X\leq 1) = P(\setb{\omega \in \Omega \colon X(\omega) \in (-\infty, 1)})$\\
\end{example}

Les v.a. permeten traslladar l'estructura d'espai de probabilitat de $(\Omega, \mathcal{A}, p)$ en $(\real, \mathcal{B})$, donant lloc a mesures que no provenen de la mesura de Lebesgue.\\

\newpage

\begin{defi}
  Siguin $(\Omega, \mathcal{A}, p)$ un espai de probabilitat i X una v.a. \\
  La \textbf{mesura de probabilitat induïda} per X és una mesura de probabilitat sobre $(\real, \mathcal{B})$ definida per
  \[
    \begin{aligned}
      p_{X} \colon \mathcal{B} &\to \real \\
      B &\mapsto p_{X} = P(\setb{\omega \in \Omega \colon X(\omega) \in B})
    \end{aligned}
  \]
  \\
\end{defi}

\begin{obs}
  $(\real, \mathcal{B},p_{X})$ és un espai de probabilitat.
\end{obs}

De teoria de la mesura, és equivalent veure que [$\forall B \in \mathcal{B}, \quad \inv{X}(B) \textit{ és de } \mathcal{A}$] a veure que [\textit{l'antiimatge de qualsevol interval} $\in \mathcal{A}$].\\\\
Per tant, per saber si una funció és una v.a. només cal veure si l'antiimatge dels intervals són de $\mathcal{A}$.\\

La següent definició dóna una funció en $\real$ que codifica molta informació de X:

\begin{defi}
  
\end{defi}



\end{document}
