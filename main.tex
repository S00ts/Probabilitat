\documentclass[12pt]{article}
 
\usepackage[margin=1in]{geometry}
\usepackage[pdftex]{hyperref}
\usepackage{amsmath,amsthm,amssymb,graphicx,mathtools,tikz,hyperref,enumerate}
\usepackage{mdframed,cleveref}
\usepackage{bbm}

\newmdenv[leftline=false,topline=false]{topright}
\let\proof\relax
\usepackage[utf8]{inputenc}
\usetikzlibrary{positioning}
\newcommand{\n}{\mathbb{N}}
\newcommand{\z}{\mathbb{Z}}
\newcommand{\q}{\mathbb{Q}}
\newcommand{\cx}{\mathbb{C}}
\newcommand{\real}{\mathbb{R}}
\newcommand{\E}{\mathbb{E}}
\newcommand{\V}{\mathbb{V}}
\newcommand{\bb}[1]{\mathbb{#1}}
\let\k\relax
\newcommand{\k}{\mathbf{k}}
\newcommand{\ita}[1]{\textit{#1}}
\newcommand\inv[1]{#1^{-1}}
\newcommand\setb[1]{\left\{#1\right\}}
\newcommand{\vbrack}[1]{\langle #1\rangle}
\newcommand{\determinant}[1]{\begin{vmatrix}#1\end{vmatrix}}
\newcommand{\abs}[1]{\left\vert #1 \right\vert}
\DeclareMathOperator{\Id}{Id}


\hypersetup{
	colorlinks,
	linkcolor=blue
}
 
 \renewcommand*\contentsname{Continguts}

\newtheoremstyle{break}% name
{}%         Space above, empty = `usual value'
{}%         Space below
{}% Body font
{}%         Indent amount (empty = no indent, \parindent = para indent)
{\bfseries}% Thm head font
{}%        Punctuation after thm head
{\newline}% Space after thm head: \newline = linebreak
{#1 #2 \normalfont #3}%         Thm head spec

\newtheoremstyle{breakthm}% name
{}%         Space above, empty = `usual value'
{}%         Space below
{}% Body font
{}%         Indent amount (empty = no indent, \parindent = para indent)
{\bfseries}% Thm head font
{}%        Punctuation after thm head
{\newline}% Space after thm head: \newline = linebreak
{#1 \normalfont #3 (#2)\addcontentsline{toc}{subsubsection}{#1 #3}}%         Thm head spec
\newtheoremstyle{normal}% name
{}%         Space above, empty = `usual value'
{}%         Space below
{}% Body font
{}%         Indent amount (empty = no indent, \parindent = para indent)
{\bfseries}% Thm head font
{}%        Punctuation after thm head
{5pt plus 1pt minus 1pt}% Space after thm head: \newline = linebreak
{#1 #2 \normalfont #3}%         Thm head spec

\theoremstyle{normal}
\newtheorem{lema}{Lema}[subsection]
\newtheorem{obs}[lema]{Observació}

\theoremstyle{break}
\newtheorem{prop}[lema]{Proposició}
\newtheorem{properties}[lema]{Propietats}
\newtheorem*{proof}{Demostració}
\newtheorem{defi}[lema]{Definició}
\newtheorem{col}[lema]{Corol·lari}
\newtheorem{ej}[lema]{Exercici}
\newtheorem{example}[lema]{Exemple}

\theoremstyle{breakthm}
\newtheorem{thm}[lema]{Teorema}
\newtheorem{defiImp}[lema]{Definició} %Pq surti al índex



\setcounter{section}{1}

 
\begin{document}
\date{}
\setlength{\parindent}{0pt}

\title{Teoria de la Probabilitat}
 
\maketitle

\tableofcontents

\setcounter{section}{0}

\newpage
\section{Espais de Probabilitat}

\subsection{Definició axiomàtica d'espai de probabilitat}

\begin{defi}
  Un \textbf{espai de probabilitat} és un espai de mesura $(\Omega, \mathcal{A}, p)$, tal que $p(\Omega) = 1$.
  \begin{itemize}
      \item $\Omega$ s'anomena \textbf{espai mostral}.
      \item $\mathcal{A}$ se l'anomena \textbf{conjunt d'esdeveniments} o \textbf{successos}.
      \item $p$ se l'anomena \textbf{funció de probabilitat}.
  \end{itemize}
\end{defi}

\begin{obs}
  $\mathcal{A} \subseteq \mathcal{P}(\Omega)$ és una $\sigma$-àlgebra: \\
  $\sigma1)$ $\varnothing \in \mathcal{A}$ \\
  $\sigma2)$ $A \in \mathcal{A} \iff \overline{A} \in \mathcal{A}$ \\
  $\sigma3)$ Si $\setb{A_{n}}_{n\geq1}$ és una seqüència de successos en $\mathcal{A} \implies \bigcup\limits_{n\geq1}A_{n} \in \mathcal{A}$
\end{obs}

\begin{obs}
  Recordem que $p$ és una mesura i, per tant: \\
  $p1)$ $p(\varnothing) = 0$ \\
  $p2)$ $\forall A \in \mathcal{A},\, p(A)\geq0$ \\
  $p3)$ Si $\setb{A_{n}}_{n\geq1}$ és una seqüència de successos en $\mathcal{A}$ disjunts 2 a 2 $(A_{i}\cap A_{j} = \varnothing \, si \, i \not =j)$, aleshores 
  \[
    p\bigg(\bigcup\limits_{n\geq1}A_{n}\bigg) = \sum_{n\geq1}p(A_{n})
  \]
\end{obs}

Vegem les primeres propietats dels espais de probabilitat:

\begin{prop}
  Per un espai de probabilitat $(\Omega, \mathcal{A}, p)$ es compleix que: \\
  (i) $A\in\mathcal{A} \implies p(\overline{A}) = 1-p(A)$ \\
  (ii) Si $A, B \in \mathcal{A},\, A \subseteq B \implies p(A)\leq p(B)$ \\
  (iii) Si $A_{1},\ldots,A_{r} \in \mathcal{A}, \, i \, A_{i}\cap A_{j} \not= \varnothing \, (i\not= j)$, aleshores $p\bigg(\bigcup\limits_{i=1}^{r}A_{i} \bigg) = \sum\limits_{i=1}^{r}p(A_{i})$ \\
  (iv) Si $A,B \in \mathcal{A}, \, A\subseteq B \implies p(B-A) = p(B)-p(A)$\\
  (v) Successions monòtones: 
  \[
    \hspace{-2.5cm}\text{a) Si } A_{1} \subseteq A_{2} \subseteq A_{3} \subseteq \ldots \subseteq A_{i} \in \mathcal{A} \implies p\bigg(\bigcup\limits_{n\geq1}A_{n}\bigg) = \lim_{n\to\infty}p(A_{n})
  \]
  \[
    \hspace{-2.5cm}\text{b) Si } A_{1} \supseteq A_{2} \supseteq A_{3} \supseteq \ldots \supseteq A_{i} \in \mathcal{A} \implies p\bigg(\bigcap\limits_{n\geq1}A_{n}\bigg) = \lim_{n\to\infty}p(A_{n})
  \]
\end{prop}

\newpage

Si ara tenim un espai de probabilitat $(\Omega, \mathcal{A},p)$, i successos $A_{1}, A_{2}, \ldots, A_{i}$ en general \underline{no} disjunts, aleshores \underline{no} és cert que  $p\bigg(\bigcup\limits_{i=1}^{r}A_{r} \bigg) = \sum\limits_{i=1}^{r}p(A_{i})$. En aquest cas, tenim la següent fita:
\begin{lema}[(Fita de la unió)] \-\\
  Siguin $A_{1}, A_{2}, \ldots, A_{r}$ successos en $(\Omega, \mathcal{A},p)$, aleshores $$p\bigg(\bigcup\limits_{i=1}^{r}A_{i} \bigg) \leq \sum\limits_{i=1}^{r}p(A_{i})$$
\end{lema}

\begin{thm}[(Desigualtats de Bonferroni)]
  Siguin $A_{1}, \ldots, A_{r}$ successos en $(\Omega, \mathcal{A}, p)$. 
  Denotem per $I\subseteq \setb{1,\ldots,r} \colon = [r]$, \\
  \[
    A_{I} = \bigcap\limits_{i\in I}A_{i}
  \]
  \[
    S_{k} = \sum_{\substack{I\subseteq[r] \\ \abs{I}=k}}p(A_{I})
  \]
  Aleshores, si: \\
  1) t és parell, $p\bigg(\bigcup\limits_{i=1}^{r} A_{i}\bigg)\geq \sum\limits_{i=1}^{t}(-1)^{i+1}\cdot S_{i}$\\\\
  2) t és senar, $p\bigg(\bigcup\limits_{i=1}^{r} A_{i}\bigg)\leq \sum\limits_{i=1}^{t}(-1)^{i+1}\cdot S_{i}$
\end{thm}

\begin{example}

\end{example}

\newpage

\subsection{Probabilitat condicionada}
\begin{defi}
  Sigui $(\Omega, \mathcal{A}, p)$ un espai de probabilitat , i $B \in \mathcal{A}$ amb $p(B) > 0$. \\
  Per $A \in \mathcal{A}$, la \textbf{probabilitat condicionada} de A amb B és:
  \[
    p(A\mid B) = \frac{p(A\cap B)}{p(B)}
  \]
\end{defi}

\begin{obs}
  $p(A \mid B)$ mesura la probabilitat de que el succés $A$ ocorri sabent que $B$ ha succeït.
\end{obs}

\begin{obs}
  Si prenem
  \[
  \begin{aligned}
      P_{B} \colon \mathcal{A} &\to \real \\
      A &\mapsto P_{B}(A) = P(A\mid B)
  \end{aligned}
  \]
  Aleshores $P_{B}$ és una funció de probabilitat sobre $\Omega, \mathcal{A}$ \\
  De fet, si definim $\mathcal{A}_{B} = \setb{A\cap B\colon A\in\mathcal{A}}$, aleshores $\mathcal{A}_{B}$ és una $\sigma$-àlgebra i $P_{B}$ també defineix una probabilitat sobre $(\Omega, \mathcal{A}_{B})$.
\end{obs}

\begin{prop}
  Siguin $A_{1}, \ldots, A_{r} \in \mathcal{A}$, tals que $p(A_{i}) > 0, \, A_{i}\cap A_{j} = \varnothing \text{ si } i \not= j$ i $\bigcup\limits_{i = 1}^{r}A_{i} = \Omega \\ \quad (\setb{A_{i}}_{i=1}^{r}$ és una partició de $\Omega)$ \\
  
  (i) \underline{Teorema de la probabilitat total}: $\forall A \in \mathcal{A}, p(A)=\sum\limits_{i=1}^{r}p(A \mid A_{i})\cdot p(A_{i})$. \\
  (ii) \underline{Fórmula de Bayes}: si $A \in \mathcal{A}, \, p(A) > 0,$
  \[
    p(A_{j} \mid A) = \frac{p(A\mid A_{j})\cdot p(A_{j})}{\sum\limits_{i=1}^{r}p(A \mid A_{i})\cdot p(A_{i})}
  \]
\end{prop}

\newpage
\section{Variables Aleatòries}
\subsection{Definició de variable aleatòria. Llei d'una v.a.}
Sigui $(\Omega, \mathcal{A}, \beta)$  un espai de probabilitat. Volem estudiar funcions de $\Omega$ amb imatge en $\real$.

\begin{defi}
  Una \textbf{variable aleatòria} és una funció $X\colon \Omega \to \real$ tal que per tot borelià B 
  $\in \mathcal{B}$, $\inv{X}(B) \in \mathcal{A}$. \\
  
  Per tant, una variable aleatòria és una funció mesurable entre els espais de mesura $(\Omega, \mathcal{A}, p)$ i $(\real, \mathcal{B}, \lambda)$.
\end{defi}

\begin{example}
  (1) Les funcions constants són variables aleatòries: \\
    \[
    \begin{aligned}
      X \colon \Omega &\to \real \\
      \omega &\mapsto c
    \end{aligned}
    \qquad \text{Si prenem } B \in \mathcal{B} \text{, } \inv{X}(B) =
    \begin{cases}
			\emptyset \quad \text{si c} \notin B\\
			\Omega \quad \text{si c} \in B
    \end{cases}
    \]
  \\
  
  (2) \textbf{Variables aleatòries indicadores}: 
    \[
    \hspace{-3.5cm}\text{Sigui A}  \in \mathcal{A}\text{, definim }\mathbbm{1}_{A}\colon \Omega \to \real \text{ on } \mathbbm{1}_{A}(\omega) = 
    \begin{cases}
			0 \quad \text{si } \omega \notin A\\
			1 \quad \text{si } \omega \in A
    \end{cases}\\
    \]
    \[
    \hspace{-3.5cm}\text{Aleshores, } B \in \mathcal{B}, \inv{\mathbbm{1}_{A}}(B) = 
    \begin{cases}
			\emptyset \quad \text{si } \setb{0, 1} \nsubseteq B\\
			A \quad \text{si } 1 \in B, \quad 0\notin B\\
			\overline{A} \quad \text{si } 1 \notin B, \quad 0 \in B\\
			\Omega \quad \text{si } \setb{0, 1} \nsubseteq B
    \end{cases}\\
    \]
    \\
    
    (3) Si X i Y són v.a., aleshores $X + Y$, $X\cdot Y$, $\abs{X}$, etc. són v.a. \\
    \- \hspace{0.5cm}En general, si $g\colon \real^{2} \to \real$ és una funció mesurable, aleshores g(X,Y) és una v.a.\\
\end{example}

Estem dient que $\forall B \in \mathcal{B}$, $\setb{\omega \in \Omega \colon X(\omega) \in B}$ és un succés i, per tant, podem calcular $P(\setb{\omega \in \Omega \colon X(\omega) \in B}) \equiv P(X \in B)$.\\

\begin{example}
  $P(X\leq 1) = P(\setb{\omega \in \Omega \colon X(\omega) \in (-\infty, 1)})$\\
\end{example}

Les v.a. permeten traslladar l'estructura d'espai de probabilitat de $(\Omega, \mathcal{A}, p)$ en $(\real, \mathcal{B})$, donant lloc a mesures que no provenen de la mesura de Lebesgue.\\

\newpage

\begin{defi}
  Siguin $(\Omega, \mathcal{A}, p)$ un espai de probabilitat i X una v.a. \\
  La \textbf{mesura de probabilitat induïda} per X és una mesura de probabilitat sobre $(\real, \mathcal{B})$ definida per
  \[
    \begin{aligned}
      p_{X} \colon \mathcal{B} &\to \real \\
      B &\mapsto p_{X} = P(\setb{\omega \in \Omega \colon X(\omega) \in B})
    \end{aligned}
  \]
  \\
\end{defi}

\begin{obs}
  $(\real, \mathcal{B},p_{X})$ és un espai de probabilitat.
\end{obs}

De teoria de la mesura, és equivalent veure que [$\forall B \in \mathcal{B}, \quad \inv{X}(B) \textit{ és de } \mathcal{A}$] a veure que [\textit{l'antiimatge de qualsevol interval} $\in \mathcal{A}$].\\\\
Per tant, per saber si una funció és una v.a. només cal veure si l'antiimatge dels intervals són de $\mathcal{A}$.\\

La següent definició dóna una funció en $\real$ que codifica molta informació de X:

\begin{defi}
  
\end{defi}

\newpage
\section{Variables Aleatòries Discretes}

\subsection{Definicions i conceptes relacionats. Funció generadora de probabilitat}

Sigui $(\Omega, \mathcal{A}, p)$ un espai de probabilitat i $X$ una variable aleatòria.

\begin{defi}
  $X$ és una \textbf{variable aleatòria discreta} si $Im(X)$ és numerable. \\
  Si $X$ és una variable aleatòria discreta, $Im(X) = \setb{x_{i}}_{i \geq 1}$. \\ 
  $X$ ve completament determinada pels valors $p(X=x_{i}) = p_{i}$.
\end{defi}

\begin{defi}
  Anomenem \textbf{funció de distribució} de X a:
  \[
    F_{X}(x)=p(X \leq x) = \sum_{x_{i}\leq x}p_{i}
  \]
\end{defi}

\begin{defi}
  Sigui $A \in \mathcal{B}$, la \textbf{mesura de probabilitat induïda sobre} $\real$ és:
  \[
    p_{X}(A) = \sum_{\substack{x_{i} \in A \\ x_{i}\in Im(X)}}p_{i}
  \]
  En particular, si $Im(X)\cap A = \varnothing$ aleshores $p_{X}(A) = 0 \implies$ obtenim una probabilitat puntual (hi ha punts de $\real$ amb probabilitat $> 0$, a diferència de la mesura de Lebesgue).
\end{defi}

\begin{defi}
  Definim l'\textbf{esperança matemàtica} com: 
  \[
    \E [X] = \int_{\Omega}Xdp \underset{\text{def. de }\int_{\Omega}}{=\joinrel=\joinrel=\joinrel=\joinrel=} \sum_{i\geq 1}x_{i}\cdot p(X=x_{i}) = \sum_{i\geq 1}x_{i}\cdot p_{i}
  \]
  Més en general, si $g(x)$ és una funció mesurable, 
  \[
    \E [g(X)] = \sum_{i\geq 1}g(x_{i})\cdot p(X=x_{i})
  \]
  En particular, 
  \[
    \E[X^{k}]=\sum_{i\geq 1}x_{i}^{k}\cdot p_{i}
  \]
  \[
    \V ar[X] = \E[X^{2}]-\E[X]^{2}=\sum_{i\geq 1}x_{i}^{2}\cdot p_{i} - \Big(\sum_{i\geq 1}x_{i}\cdot p_{i} \Big)^{2}
  \]
\end{defi}

\begin{defi}
  Prenem $X$, $Y$ variables aleatòries discretes amb $Im(X), \, Im(Y)$ numerables. \\
  El \textbf{vector de variables aleatòries} $(X,Y)$ ve completament caracteritzat per: 
  \[
    p_{(X,Y)}(x_{i},y_{j}) = p(X=x_{i}, Y=y_{j}) \qquad (x_{i}\in Im(X); y_{i}\in Im(Y))
  \]
\end{defi}

\newpage
Aleshores, tenim les següents propietats: \\\\
(i) $X$ i $Y$ són independents $\iff \forall x_{i}\in Im(X), \, \forall y_{j}\in Im(Y), \, p_{(X, Y)}(x_{i},y_{j}) = p_{X}(x_{i})\cdot p_{Y}(y_{j})$ \\\\
(ii) Si $X, \, Y$ són independents, $\E[X\cdot Y] = \E[X]\cdot \E[Y]$ \\\\

Sigui $X$ una variable aleatòria discreta amb $Im(X)\subseteq \n_{\geq0}$. En aquesta situació, tenim la següent definició:

\begin{defi}
  La \textbf{funció generadora de probabilitat associada a $X$} és: 
  \[
    G_{X}(z) = \sum_{i\geq 0}p(X=i)\cdot z^{i} = \sum_{i\geq 0}p_{i}\cdot z^{i}
  \]
  
  Una funció generadora de probabilitat és un objecte formal. En particular, $G_{X}(z)=\E[z^{X}]$.
\end{defi}

Si ens mirem les funcions generadores de probabilitat com funcions (en $\cx$), aleshores compleixen el següent:

\begin{prop}
  \begin{enumerate}
      \item []
      \item $G_{X}(0) = p(X=0)$
      \item $G_{X}(1) = 1$. A més, si $\abs{z}\leq 1 \, (z\in \cx) \implies \abs{G_{X}(z)}\leq 1$ \\
      Per tant, $G_{X}(z)$ (com a sèrie de potències) té radi de convergència $\geq 1$.
      \item $\E[X] = \dfrac{d}{dz}G_{X}(z)_{\big|z=1}$
      \item Més en general, $\E[(X)_{k}] = \dfrac{d^{k}}{dz^{k}}G_{X}(z)_{\big|z=1}$ \\
      En particular, $\V ar[X] = G_{X}''(1) + G_{X}'(1) - (G_{X}'(1))^{2}$
      \item Si $z\in \real, \, z \in [0,1]$, aleshores $G_{X}(z)$ és una funció creixent.
  \end{enumerate}
\end{prop}

La propietat fonamental de les funcions generadores de probabilitat és que permeten estudiar fàcilment sumes de variables aleatòries independents.

\begin{prop}
  Siguin $X, Y$ variables aleatòries discretes independents amb imatge en $\n$, i amb funcions generadores de probabilitat $G_{X}(z), \, G_{Y}(z)$. Aleshores: 
  \[
    G_{X+Y}(z) = G_{X}(z)\cdot G_{Y}(z)
  \]
\end{prop}

\begin{col}
  Si $X_{1}, \ldots, X_{N}$ són variables aleatòries discretes independents amb imatge en $\n$, aleshores: 
  \[
    G_{X_{1}+\ldots+X_{N}}(z) = \prod_{i=1}^{N} G_{X_{i}}(z)
  \]
\end{col}

\subsection{Models de variables aleatòries discretes}

Seguidament veurem famílies importants de variables aleatòries discretes: \\

1) \underline{Variable aleatòria Bernoulli}: $X\sim B(p)$ \quad (un sol paràmetre) \\
Modela el llançament d'una moneda amb probabilitat d'èxit igual a p: 
\[
  p(X=1) = p, \quad p(X=0) = 1-p
\]
\begin{itemize}
    \item $G_{X}(z) = p\cdot z + (1-p)$ \quad (és una funció entera)
    \item $\E[X] = p$
    \item $\V ar[X] = p\cdot(1-p)$
\end{itemize}

\vspace{0.5cm}

2) \underline{Binomial}: $X\sim Bin(p,n)$ (o bé $B(p,n)$) \\
Nombre d'èxits al tirar una moneda $n$ vegades. L'èxit individual té probabilitat $p$, i cada tirada de la moneda és independent de la resta. 
\[
  \implies X=Y_{1}+\ldots+Y_{n} \text{ on } Y_{i}\sim B(p)
\]
\[
  p(X=i) = \binom{n}{i}\cdot p^{i}\cdot (1-p)^{n-i} \qquad i = 0,\ldots,n
\]
\begin{itemize}
    \item $G_{X}(z) = (p\cdot z + (1-p))^{n}$ \quad (és una funció entera)
    \item $\E[X] = n\cdot p$
    \item $\V ar[X] = n\cdot p\cdot (1-p)$
\end{itemize}

\vspace{0.5cm}

3) \underline{Poisson}: $X \sim Po(\lambda)$
\[
  p(X=k) = \frac{1}{k!}\cdot \lambda^{k}\cdot e^{-\lambda} \quad (k\in \n_{\geq 0})
\]
\begin{itemize}
    \item $G_{X}(z) = e^{\lambda(z-1)}$ \quad (és una funció entera)
    \item $\E[X] = \lambda$
    \item $\V ar[X] = \lambda$
\end{itemize}

\vspace{0.5cm}

4) \underline{Uniforme}: $X\sim U(N)$ \quad (on $Im(X) = \setb{1,2,\ldots, N}$) \\
\[
  p(X=k) = \frac{1}{n}
\]
\begin{itemize}
    \item $G_{X} = \dfrac{1}{N}\cdot \dfrac{1-z^{N}}{1-z}\cdot z$ \quad (sing. evitable en z = 1)
    \item $\E[X]=\dfrac{N+1}{2}$
    \item $\V ar[X] = \dfrac{N^{2}-1}{12}$
\end{itemize}

\newpage

5) \underline{Geomètrica}: $X\sim Geom(p) \quad \big(p\in(0,1)\big)$ \\
Nombre de tirades d'una moneda (amb probabilitat d'èxit $= p$) fins aconseguir el primer èxit.
\[
  p(X=k) = (1-p)^{k-1}\cdot p \qquad k = 1,2,\ldots
\]
\begin{itemize}
    \item $G_{X}(z) = \dfrac{p\cdot z}{1-(1-p)\cdot z}$ \quad (té un pol en $z=\dfrac{1}{1-p}$)
    \item $\E[X] = \dfrac{1}{p}$
    \item $\V ar[X] = \dfrac{1-p}{p^{2}}$
\end{itemize}

\vspace{0.5cm}

6) \underline{Binomial negativa}: $X\sim BinN(r, p)$ \\
Variable aleatòria que compta el nombre de tirades necessàries per aconseguir $r$ èxits. $\left(Im(X)=\setb{r,r+1,r+2, \ldots}\right)$
\[
  p(X=k) = \binom{k-1}{r-1}\cdot p^{r}\cdot(1-p)^{k-r}
\]
Podem interpretar una binomial negativa com una suma de geomètriques independents.
\begin{itemize}
    \item $G_{X}(z) = \left( \dfrac{p\cdot z}{1-(1-p)\cdot z} \right)^{r}$
    \item $\E[X] = \dfrac{r}{p}$
    \item $\V ar[X] = r\cdot \dfrac{1-p}{p^2}$
\end{itemize}

\newpage

\subsection{Distribucions condicionades i esperança condicionada}
Siguin $X,Y$ dues variables aleatòries discretes. Volem definir la noció de condicionar una variable aleatòria a l'altra (`` $Y\mid X$ '').

\begin{defi}
  Donat $x$ amb $p(X=x)>0$, diem que la \textbf{funció de probabilitat condicionada} de $Y$ amb $X$ és, per aquest valor de $x$:
  \[
    p_{Y\mid X}(y,x) = p(Y=y\mid X=x)
  \]
\end{defi}

\begin{defi}
  Amb les mateixes condicions que abans, la \textbf{funció de distribució de probabilitat condicionada} és:
  \[
    F_{Y\mid X}(y,x) = p(Y\leq y \mid X \leq x)
  \]
\end{defi}

\begin{obs}
  Si $X$ i $Y$ són independents, $X\mid Y=y \sim X$
\end{obs}

\begin{defi}
  L'\textbf{esperança condicionada} de $Y$ a $X=x$ és: 
  \[
    \psi(x) = \E[Y\mid X=x] = \sum_{y\in Im(Y)}(y\cdot p_{Y\mid X})(y,x)
  \]
  Amb aquesta definició estem definint una variable aleatòria $E[Y\mid X]$ que pren el valor $\psi(x)$ amb probabilitat $p(X=x)$.
\end{defi}

\begin{prop}
  $\E\big[\E[Y\mid X]\big] = \E[Y]$
\end{prop}

\begin{obs}
  $\E[Y] = \sum\limits_{x\in Im(X)}\E[Y\mid X=x]\cdot p(X=x)$
\end{obs}

\begin{obs}
  $\E[Y\mid X]$ és la millor aproximació de $Y$ com a funció de X.
\end{obs}



\end{document}
