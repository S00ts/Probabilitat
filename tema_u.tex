\section{Espais de Probabilitat}

\subsection{Definició axiomàtica d'espai de probabilitat}

\begin{defi}
  Un \textbf{espai de probabilitat} és un espai de mesura $(\Omega, \mathcal{A}, p)$, tal que $p(\Omega) = 1$.
  \begin{itemize}
      \item $\Omega$ s'anomena \textbf{espai mostral}.
      \item $\mathcal{A}$ se l'anomena \textbf{conjunt d'esdeveniments} o \textbf{successos}.
      \item $p$ se l'anomena \textbf{funció de probabilitat}.
  \end{itemize}
\end{defi}

\begin{obs}
  $\mathcal{A} \subseteq \mathcal{P}(\Omega)$ és una $\sigma$-àlgebra: \\
  $\sigma1)$ $\varnothing \in \mathcal{A}$ \\
  $\sigma2)$ $A \in \mathcal{A} \iff \overline{A} \in \mathcal{A}$ \\
  $\sigma3)$ Si $\setb{A_{n}}_{n\geq1}$ és una seqüència de successos en $\mathcal{A} \implies \bigcup\limits_{n\geq1}A_{n} \in \mathcal{A}$
\end{obs}

\begin{obs}
  Recordem que $p$ és una mesura i, per tant: \\
  $p1)$ $p(\varnothing) = 0$ \\
  $p2)$ $\forall A \in \mathcal{A},\, p(A)\geq0$ \\
  $p3)$ Si $\setb{A_{n}}_{n\geq1}$ és una seqüència de successos en $\mathcal{A}$ disjunts 2 a 2 $(A_{i}\cap A_{j} = \varnothing \, si \, i \not =j)$, aleshores 
  \[
    p\bigg(\bigcup\limits_{n\geq1}A_{n}\bigg) = \sum_{n\geq1}p(A_{n})
  \]
\end{obs}

Vegem les primeres propietats dels espais de probabilitat:

\begin{prop}
  Per un espai de probabilitat $(\Omega, \mathcal{A}, p)$ es compleix que: \\
  (i) $A\in\mathcal{A} \implies p(\overline{A}) = 1-p(A)$ \\
  (ii) Si $A, B \in \mathcal{A},\, A \subseteq B \implies p(A)\leq p(B)$ \\
  (iii) Si $A_{1},\ldots,A_{r} \in \mathcal{A}, \, i \, A_{i}\cap A_{j} \not= \varnothing \, (i\not= j)$, aleshores $p\bigg(\bigcup\limits_{i=1}^{r}A_{i} \bigg) = \sum\limits_{i=1}^{r}p(A_{i})$ \\
  (iv) Si $A,B \in \mathcal{A}, \, A\subseteq B \implies p(B-A) = p(B)-p(A)$\\
  (v) Successions monòtones: 
  \[
    \hspace{-2.5cm}\text{a) Si } A_{1} \subseteq A_{2} \subseteq A_{3} \subseteq \ldots \subseteq A_{i} \in \mathcal{A} \implies p\bigg(\bigcup\limits_{n\geq1}A_{n}\bigg) = \lim_{n\to\infty}p(A_{n})
  \]
  \[
    \hspace{-2.5cm}\text{b) Si } A_{1} \supseteq A_{2} \supseteq A_{3} \supseteq \ldots \supseteq A_{i} \in \mathcal{A} \implies p\bigg(\bigcap\limits_{n\geq1}A_{n}\bigg) = \lim_{n\to\infty}p(A_{n})
  \]
\end{prop}

\newpage

Si ara tenim un espai de probabilitat $(\Omega, \mathcal{A},p)$, i successos $A_{1}, A_{2}, \ldots, A_{i}$ en general \underline{no} disjunts, aleshores \underline{no} és cert que  $p\bigg(\bigcup\limits_{i=1}^{r}A_{r} \bigg) = \sum\limits_{i=1}^{r}p(A_{i})$. En aquest cas, tenim la següent fita:
\begin{lema}[(Fita de la unió)] \-\\
  Siguin $A_{1}, A_{2}, \ldots, A_{r}$ successos en $(\Omega, \mathcal{A},p)$, aleshores $$p\bigg(\bigcup\limits_{i=1}^{r}A_{i} \bigg) \leq \sum\limits_{i=1}^{r}p(A_{i})$$
\end{lema}

\begin{thm}[(Desigualtats de Bonferroni)]
  Siguin $A_{1}, \ldots, A_{r}$ successos en $(\Omega, \mathcal{A}, p)$. 
  Denotem per $I\subseteq \setb{1,\ldots,r} \colon = [r]$, \\
  \[
    A_{I} = \bigcap\limits_{i\in I}A_{i}
  \]
  \[
    S_{k} = \sum_{\substack{I\subseteq[r] \\ \abs{I}=k}}p(A_{I})
  \]
  Aleshores, si: \\
  1) t és parell, $p\bigg(\bigcup\limits_{i=1}^{r} A_{i}\bigg)\geq \sum\limits_{i=1}^{t}(-1)^{i+1}\cdot S_{i}$\\\\
  2) t és senar, $p\bigg(\bigcup\limits_{i=1}^{r} A_{i}\bigg)\leq \sum\limits_{i=1}^{t}(-1)^{i+1}\cdot S_{i}$
\end{thm}

\begin{example}

\end{example}

\newpage

\subsection{Probabilitat condicionada}
\begin{defi}
  Sigui $(\Omega, \mathcal{A}, p)$ un espai de probabilitat , i $B \in \mathcal{A}$ amb $p(B) > 0$. \\
  Per $A \in \mathcal{A}$, la \textbf{probabilitat condicionada} de A amb B és:
  \[
    p(A\mid B) = \frac{p(A\cap B)}{p(B)}
  \]
\end{defi}

\begin{obs}
  $p(A \mid B)$ mesura la probabilitat de que el succés $A$ ocorri sabent que $B$ ha succeït.
\end{obs}

\begin{obs}
  Si prenem
  \[
  \begin{aligned}
      P_{B} \colon \mathcal{A} &\to \real \\
      A &\mapsto P_{B}(A) = P(A\mid B)
  \end{aligned}
  \]
  Aleshores $P_{B}$ és una funció de probabilitat sobre $\Omega, \mathcal{A}$ \\
  De fet, si definim $\mathcal{A}_{B} = \setb{A\cap B\colon A\in\mathcal{A}}$, aleshores $\mathcal{A}_{B}$ és una $\sigma$-àlgebra i $P_{B}$ també defineix una probabilitat sobre $(\Omega, \mathcal{A}_{B})$.
\end{obs}

\begin{prop}
  Siguin $A_{1}, \ldots, A_{r} \in \mathcal{A}$, tals que $p(A_{i}) > 0, \, A_{i}\cap A_{j} = \varnothing \text{ si } i \not= j$ i $\bigcup\limits_{i = 1}^{r}A_{i} = \Omega \\ \quad (\setb{A_{i}}_{i=1}^{r}$ és una partició de $\Omega)$ \\
  
  (i) \underline{Teorema de la probabilitat total}: $\forall A \in \mathcal{A}, p(A)=\sum\limits_{i=1}^{r}p(A \mid A_{i})\cdot p(A_{i})$. \\
  (ii) \underline{Fórmula de Bayes}: si $A \in \mathcal{A}, \, p(A) > 0,$
  \[
    p(A_{j} \mid A) = \frac{p(A\mid A_{j})\cdot p(A_{j})}{\sum\limits_{i=1}^{r}p(A \mid A_{i})\cdot p(A_{i})}
  \]
\end{prop}

Molts cops serà més senzill calcular probabilitats condicionades. Vegem un exemple:
\begin{example}[(La ruïna del jugador)]
  Partim d'un capital de $k$ unitats $(k \geq 0)$ i volem aconseguir un capital de $N$ unitats $(N\geq k)$ de la següent forma: \\
  Llancem una moneda equilibrada, guanyant (surt cara) o perdent (surt creu) una unitat amb probabilitat $\frac{1}{2}$. El joc acaba si: 
  \begin{enumerate}
      \item Ens quedem sense capital.
      \item Assolim un capital igual a N
  \end{enumerate}
  Calcularem la probabilitat de perdre. \\
  No és bona idea intentar codificar tots els casos possibles i sumar les seves probabilitats \\ (nº tirades $\to \infty$). \\\\
  
  Anem a resoldre el problema usant el teorema de la probabilitat total. \\
  $A_{k} = $ "el jugador, amb capital inicial igual a $k$, s'arruïna" (Volem calcular $p(A_{k})$). \\
  Escrivim $p(A_{k}) = p_{k}$. Aleshores $p_{0} = 1, \, p_{N} = 0$. \\
  Definim $B =$ "la primera tirada de la moneda és cara". Aleshores $B$ i $\overline{B}$ defineixen una partició de $\Omega$. \\
  Usem el teorema de probabilitat total:
  \[
    \begin{rcases}
        p(A_{k}) &= p(A_{k} \mid B)p(B) + p(A_{k}\mid \overline{B})p(\overline{B}) \\
        &= p(A_{k}\mid B)\cdot\frac{1}{2} + p(A_{k}\mid\overline{B})\cdot\frac{1}{2} \\
        &= \frac{1}{2} \cdot\Big( \underbrace{p(A_{k}\mid B)}_{p(A_{k+1})} + \underbrace{p(A_{k}\mid\overline{B})}_{p(A_{k-1})}\Big)
    \end{rcases} p_{k} = \frac{1}{2}(p_{k+1} + p_{k-1}); \quad p_{0} = 1, \, p_{N} = 0
  \]
  
  Resolent la recurrència tenim: 
  \[
    p_{k} = 1 - \frac{k}{N}
  \]
  Si $k \in o(N)$, aleshores assimptòticament el jugador s'acabarà arruïnant.
\end{example}

\subsection{Independència}

\begin{defi}
  Dos successos $A$ i $B$ (del mateix espai $(\Omega, \mathcal{A}, p)$) són \textbf{independents} si $p(A\cap B) = p(A)\cdot p(B)$.
\end{defi}

\begin{obs}
  \begin{itemize}
      \item []
      \item Si $p(B) > 0$, $A$ i $B$ independents $\iff p(A\mid B)=p(A)$
      \item El successos $\varnothing$ i $\Omega$ són independents amb qualsevol altre succés B
      \item Si $A$ i $B$ són independents, $A$ i $\overline{B}$ també. (De fet tenim que $A$ i $B$ són independents $\iff \overline{A}$ i $\overline{B}$ són independents).
  \end{itemize}
\end{obs}

\begin{defi}
  Donada una família de successos $\setb{A_{i}}_{i\in I}$, es diu que és \textbf{independent} si:
  \[
    \forall J \subseteq I \text{ amb } \abs{J} < \infty \text{, es té: } p\bigg(\bigcap\limits_{j\in J}A_{j}\bigg) = \prod_{j\in J}p(A_{j}) 
  \]
\end{defi}

\begin{obs}
  Donats $\setb{A_{i}}_{i = 1, \ldots, r}$ successos, poden ser independents dos a dos però no com a conjunt.
\end{obs}

\newpage

\subsection{Espai producte}

Donats dos espais de probabilitat $(\Omega_{1}, \mathcal{A}_{1}, p_{1})$ i $(\Omega_{2}, \mathcal{A}_{2}, p_{2})$, volem combinar-los en un sol espai amb espai mostral $\Omega_{1}\bigtimes\Omega_{2}$. \\
Això ja ho hem fet en el cas discret, però en general:
\begin{itemize}
    \item Prenem com a espai mostral $\Omega_{1}\bigtimes\Omega_{2}$
    \item Prenem com a $\sigma$-àlgebra $\sigma(\mathcal{A}_{1}\bigtimes\mathcal{A}_{2})$ (la més petita $\sigma$-àlgebra que conté $\mathcal{A}_{1}\bigtimes\mathcal{A}_{2}$).
    \item Com definim $p$ a $(\Omega_{1}\bigtimes\Omega_{2}, \mathcal{A}_{1}\bigtimes\mathcal{A}_{2})$? \\ Voldríem que $p(A_{1}\bigtimes A_{2}) = p(A_{1})\cdot p(A_{2}) \quad \forall A_{i} \in \mathcal{A}_{i}$. Per definir $p$ necessitarem el següent teorema:
\end{itemize}

\begin{thm}[(d'extensió o de Carathéodory)]
  Sigui $p_{0}$ una funció de probabilitat en una àlgebra $\mathcal{A}_{0}$. Sigui $\mathcal{A} = \sigma(\mathcal{A}_{0})$. \\
  Aleshores $p_{0}$ es pot extendre a una funció de probabilitat $p$ sobre $\mathcal{A}$ que coincideix amb $p_{0}$ en $\mathcal{A}_{0}$. A més, $p$ és única.
\end{thm}

En el nostre cas, quina àlgebra agafem? 
\[
    \underbrace{(\mathcal{A}_{1}\bigtimes\mathcal{A}_{2})^{*}}_{\mathcal{A}_{0}} = \text{ totes les unions finites d'elements de } \mathcal{A}_{1}\bigtimes\mathcal{A}_{2}.
\]
Es comprova que $\sigma((\mathcal{A}_{1}\bigtimes\mathcal{A}_{2})^{*}) = \sigma(\mathcal{A}_{1}\bigtimes\mathcal{A}_{2})$ i apliquem el teorema.
