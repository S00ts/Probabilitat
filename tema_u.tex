\section{Espais de Probabilitat}

\subsection{Definició axiomàtica d'espai de probabilitat}

\begin{defi}
  Un \textbf{espai de probabilitat} és un espai de mesura $(\Omega, \mathcal{A}, p)$, tal que $p(\Omega) = 1$.
  \begin{itemize}
      \item $\Omega$ s'anomena \textbf{espai mostral}.
      \item $\mathcal{A}$ se l'anomena \textbf{conjunt d'esdeveniments} o \textbf{successos}.
      \item $p$ se l'anomena \textbf{funció de probabilitat}.
  \end{itemize}
\end{defi}

\begin{obs}
  $\mathcal{A} \subseteq \mathcal{P}(\Omega)$ és una $\sigma$-àlgebra: \\
  $\sigma1)$ $\varnothing \in \mathcal{A}$ \\
  $\sigma2)$ $A \in \mathcal{A} \iff \overline{A} \in \mathcal{A}$ \\
  $\sigma3)$ Si $\setb{A_{n}}_{n\geq1}$ és una seqüència de successos en $\mathcal{A} \implies \bigcup\limits_{n\geq1}A_{n} \in \mathcal{A}$
\end{obs}

\begin{obs}
  Recordem que $p$ és una mesura i, per tant: \\
  $p1)$ $p(\varnothing) = 0$ \\
  $p2)$ $\forall A \in \mathcal{A},\, p(A)\geq0$ \\
  $p3)$ Si $\setb{A_{n}}_{n\geq1}$ és una seqüència de successos en $\mathcal{A}$ disjunts 2 a 2 $(A_{i}\cap A_{j} = \varnothing \, si \, i \not =j)$, aleshores 
  \[
    p\bigg(\bigcup\limits_{n\geq1}A_{n}\bigg) = \sum_{n\geq1}p(A_{n})
  \]
\end{obs}

Vegem les primeres propietats dels espais de probabilitat:

\begin{prop}
  Per un espai de probabilitat $(\Omega, \mathcal{A}, p)$ es compleix que: \\
  (i) $A\in\mathcal{A} \implies p(\overline{A}) = 1-p(A)$ \\
  (ii) Si $A, B \in \mathcal{A},\, A \subseteq B \implies p(A)\leq p(B)$ \\
  (iii) Si $A_{1},\ldots,A_{r} \in \mathcal{A}, \, i \, A_{i}\cap A_{j} \not= \varnothing \, (i\not= j)$, aleshores $p\bigg(\bigcup\limits_{i=1}^{r}A_{i} \bigg) = \sum\limits_{i=1}^{r}p(A_{i})$ \\
  (iv) Si $A,B \in \mathcal{A}, \, A\subseteq B \implies p(B-A) = p(B)-p(A)$\\
  (v) Successions monòtones: 
  \[
    \hspace{-2.5cm}\text{a) Si } A_{1} \subseteq A_{2} \subseteq A_{3} \subseteq \ldots \subseteq A_{i} \in \mathcal{A} \implies p\bigg(\bigcup\limits_{n\geq1}A_{n}\bigg) = \lim_{n\to\infty}p(A_{n})
  \]
  \[
    \hspace{-2.5cm}\text{b) Si } A_{1} \supseteq A_{2} \supseteq A_{3} \supseteq \ldots \supseteq A_{i} \in \mathcal{A} \implies p\bigg(\bigcap\limits_{n\geq1}A_{n}\bigg) = \lim_{n\to\infty}p(A_{n})
  \]
\end{prop}

\newpage

Si ara tenim un espai de probabilitat $(\Omega, \mathcal{A},p)$, i successos $A_{1}, A_{2}, \ldots, A_{i}$ en general \underline{no} disjunts, aleshores \underline{no} és cert que  $p\bigg(\bigcup\limits_{i=1}^{r}A_{r} \bigg) = \sum\limits_{i=1}^{r}p(A_{i})$. En aquest cas, tenim la següent fita:
\begin{lema}[(Fita de la unió)] \-\\
  Siguin $A_{1}, A_{2}, \ldots, A_{r}$ successos en $(\Omega, \mathcal{A},p)$, aleshores $$p\bigg(\bigcup\limits_{i=1}^{r}A_{i} \bigg) \leq \sum\limits_{i=1}^{r}p(A_{i})$$
\end{lema}

\begin{thm}[(Desigualtats de Bonferroni)]
  Siguin $A_{1}, \ldots, A_{r}$ successos en $(\Omega, \mathcal{A}, p)$. 
  Denotem per $I\subseteq \setb{1,\ldots,r} \colon = [r]$, \\
  \[
    A_{I} = \bigcap\limits_{i\in I}A_{i}
  \]
  \[
    S_{k} = \sum_{\substack{I\subseteq[r] \\ \abs{I}=k}}p(A_{I})
  \]
  Aleshores, si: \\
  1) t és parell, $p\bigg(\bigcup\limits_{i=1}^{r} A_{i}\bigg)\geq \sum\limits_{i=1}^{t}(-1)^{i+1}\cdot S_{i}$\\\\
  2) t és senar, $p\bigg(\bigcup\limits_{i=1}^{r} A_{i}\bigg)\leq \sum\limits_{i=1}^{t}(-1)^{i+1}\cdot S_{i}$
\end{thm}

\begin{example}

\end{example}

\newpage

\subsection{Probabilitat condicionada}
\begin{defi}
  Sigui $(\Omega, \mathcal{A}, p)$ un espai de probabilitat , i $B \in \mathcal{A}$ amb $p(B) > 0$. \\
  Per $A \in \mathcal{A}$, la \textbf{probabilitat condicionada} de A amb B és:
  \[
    p(A\mid B) = \frac{p(A\cap B)}{p(B)}
  \]
\end{defi}

\begin{obs}
  $p(A \mid B)$ mesura la probabilitat de que el succés $A$ ocorri sabent que $B$ ha succeït.
\end{obs}

\begin{obs}
  Si prenem
  \[
  \begin{aligned}
      P_{B} \colon \mathcal{A} &\to \real \\
      A &\mapsto P_{B}(A) = P(A\mid B)
  \end{aligned}
  \]
  Aleshores $P_{B}$ és una funció de probabilitat sobre $\Omega, \mathcal{A}$ \\
  De fet, si definim $\mathcal{A}_{B} = \setb{A\cap B\colon A\in\mathcal{A}}$, aleshores $\mathcal{A}_{B}$ és una $\sigma$-àlgebra i $P_{B}$ també defineix una probabilitat sobre $(\Omega, \mathcal{A}_{B})$.
\end{obs}

\begin{prop}
  Siguin $A_{1}, \ldots, A_{r} \in \mathcal{A}$, tals que $p(A_{i}) > 0, \, A_{i}\cap A_{j} = \varnothing \text{ si } i \not= j$ i $\bigcup\limits_{i = 1}^{r}A_{i} = \Omega \\ \quad (\setb{A_{i}}_{i=1}^{r}$ és una partició de $\Omega)$ \\
  
  (i) \underline{Teorema de la probabilitat total}: $\forall A \in \mathcal{A}, p(A)=\sum\limits_{i=1}^{r}p(A \mid A_{i})\cdot p(A_{i})$. \\
  (ii) \underline{Fórmula de Bayes}: si $A \in \mathcal{A}, \, p(A) > 0,$
  \[
    p(A_{j} \mid A) = \frac{p(A\mid A_{j})\cdot p(A_{j})}{\sum\limits_{i=1}^{r}p(A \mid A_{i})\cdot p(A_{i})}
  \]
\end{prop}
