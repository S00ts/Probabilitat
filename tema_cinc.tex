\section{Funcions característiques i famílies exponencials}

\subsection{Funció generadora de moments i funció característica}

\begin{defi}
  Sigui $(\Omega, \mathcal{A}, p)$ un espai de probabilitat i $X$ una variable aleatòria, la \textbf{funció generadora de moments} de $X$ és $M_{X}(s)$, definida per:
  \[
    M_{X}(s) = \E[e^{sX}] \quad \text{ on s $\in \cx \quad$ (No té per què estar definida)}
  \]
\end{defi}

\begin{properties}
  \begin{enumerate}
      \item $M_{X}(0) = 1$
      \item Si $\E[X^{k}] < \infty \forall k \geq 1, aleshores \E[X^{k}]=M_{X}^{(k)}(0)$
      \item Si $Y=aX+b$,
      \[
        M_{Y}(s) = \E[e^{s\cdot(aX+b)}] = e^{sb}\E[e^{saX}] = e^{sb}\cdot M_{X}(as)
      \]
      \item Si $X,Y$ són indep. $\implies M_{X+Y}(s) = \E[e^{s\cdot(X+Y)}] = \E[e^{sX}]\cdot\E[e^{sY}] = M_{X}(s)\cdot M_{Y}(s)$
  \end{enumerate}
\end{properties}
\-\\
Aquesta definició particularitza en el cas discret i en el cas continuu de la següent manera:
\begin{itemize}
    \item Cas discret: $\E[e^{sX}] = \sum\limits_{x\in Im(X)}e^{sx}p(X=x)$
    \item Cas continuu: $\E[e^{sx}] = \int_{\real}e^{sx}f_{X}(x) \, dx$
\end{itemize}

\begin{example}
  \begin{itemize}
  \item[]
      \item $X \sim Ber(p) \implies M_{X}(s) = e^{s\cdot0}\cdot(1-p) + e^{s\cdot1}\cdot p = p\cdot e^{s} + (1-p)$ \\\\
      $M_{X}(s)$ és una funció entera $\implies$ podem calcular tots els moments: $\E[X^{k}] = p$ \\
      \item Si $Y \sim Bin(N,p) \implies Y=X_{1}+\ldots+X_{N} \quad \text{on } X_{i}\sim Ber(p), \, \setb{X_{i}}_{i=1}^{N} \text{ independents}$
      \[
        M_{Y}(s) = (p\cdot e^{s} + (1-p))^{N}
      \]
      \item $X\sim Exp(\lambda) \quad (\lambda > 0)$
      \[
        M_{X}(s) = \int_{\real}e^{sx}\cdot\lambda e^{-\lambda x} \, dx = \lambda\int_{-\infty}^{+\infty} e^{(s-\lambda)x} \, dx = \lambda\cdot\frac{e^{(s-\lambda)x}}{s-\lambda}\bigg|_{-\infty}^{+\infty}
      \]
      (No està definida per alguns valors de $s$)
      \item $X \sim Cauchy. \, X$ té funció de densitat $\dfrac{1}{\pi(1+x^{2})}$
      \[
        \E[e^{sX}] = \int_{\real}\frac{e^{sx}}{\pi(1+x^{2})}\, dx \quad \text{(No té sentit si $Re(s)>0$)}
      \]
  \end{itemize}
  
\end{example}

\newpage
  
Estem veient que la funció generadora de moments no convergeix en la majoria de casos. Això és degut a que $s\in\cx$. \\
Per a solucionar aquest problema, prenem $s=it$ ($t\in\real$).

\begin{defi}
  La funció $\E[e^{itX}] = M_{X}(it) = \Phi_{X}(t)$ és la \textbf{funció característica} de X.
\end{defi}

\begin{obs}
  Totes les propietats de la funció generadora de moments es compleixen per les funcions característiques:
  \begin{enumerate}
      \item  $\Phi_{X}(0) = 1$
      \item $\E[X^{k}] = M_{X}^{(k)}(0) = \dfrac{d^{k}}{ds^{k}}M_{X}(s)\bigg|_{s=0} = \dfrac{1}{(i)^{k}}\cdot \Phi_{X}^{(k)}(0)$
      \item $Y = aX+b \implies \Phi_{Y}(t) = e^{ibt}\cdot \Phi_{X}(at)$
      \item $X,Y$ independents $\implies \Phi_{X+Y}(t)=\Phi_{X}(t)\cdot\Phi_{Y}(t)$
  \end{enumerate}
\end{obs}

\begin{example}
  \begin{itemize}
      \item[]
      \item $X\sim Exp(\lambda) \quad (\text{amb } \lambda > 0)$. Té funció de densitat $f_{X}(x) = \lambda e^{-\lambda x}\cdot \mathbbm{1}_{[0, \infty)}(x)$
      \[
        \Phi_{X}(t) = \int_{0}^{\infty}e^{itx}\cdot\lambda e^{-\lambda x} \, dx = \lambda\int_{0}^{\infty}e^{(it-\lambda)x} \, dx = \lambda\frac{e^{(it-\lambda)x}}{it-\lambda}\bigg|_{0}^{\infty} = \frac{-\lambda}{it-\lambda}
      \]
      (Està definida per tot valor de t)
      \item $X\sim Cauchy$
      \[
        \Phi_{X}(t) = \int_{-\infty}^{+\infty}\frac{e^{itx}}{\pi(1+x^{2})} \, dx = e^{-\abs{t}}
      \]
      S'observa que la variable aleatòria $Cauchy$ \underline{no} té moment, però si que té funció característica (No es pot derivar $\Phi_{X}(t)$ en $t=0$)
      \item $X\sim N(0,1)$ : $\Phi_{X}(t) = e^{\frac{-t^{2}}{2}}$ \\\\
      Com a conseqüència, si $X\sim N(\mu, \sigma^{2})$, aleshores $X=\sigma Y + \mu$ (on $Y\sim N(0,1)$)
      \[
        \implies \Phi_{X}(t) = e^{i\mu t - \frac{1}{2}\sigma^{2}t^{2}}
      \]
      
      \item $X\sim Geom(p)$: $\Phi_{X}(t) = \dfrac{p}{e^{it}-(1-p)}$
      \item $X\sim Pois(\lambda)$: $\Phi_{X}(t) = e^{\lambda(e^{it}-1)}$
      \item $X\sim U(0,1)$: $\Phi_{X}(t) = \dfrac{e^{it}-1}{it} \quad$ (Té una singularitat evitable en t=0)
  \end{itemize}
\end{example}

\subsubsection{Propietats generals de les funcions característiques}
\begin{prop}
  Sigui $X$ una variable aleatòria amb funció característica $\Phi_{X}(t)$. Aleshores:
  \begin{enumerate}
      \item $\Phi_{X}(0)=1$ i $\abs{\Phi_{X}(t)}\leq 1 \, \forall t\in\real$
      \item $\Phi_{X}(t)$ és uniformement contínua en $\real$.
      \item $\forall t_{1},\ldots,t_{n} \in \real, \forall z_{1},\ldots,z_{n}\in\cx$, $\sum\limits_{j,k}\Phi_{X}(t_{j}-t_{k})z_{j}\cdot \overline{z}_{k}\geq0$
  \end{enumerate}
\end{prop}

\begin{thm}[(d'inversió)]
  Sigui $X$ una variable aleatòria que indueix una probabilitat $p_{X}$ sobre $\real$ i funció característica $\Phi_{X}(t)$. Aleshores $\forall a,b\in \real \, (a<b)$,
  \[
    \underbrace{p_{X}\big((a,b)\big)}_{p(a<X<b)} + \frac{1}{2}\big(p_{X}(\setb{a})+p_{X}(\setb{b}) \big) = \lim_{T\to+\infty}\frac{1}{2\pi}\int_{-T}^{T}\frac{e^{-ita}-e^{-itb}}{it}\cdot\Phi_{X}(t) \, dt
  \]
\end{thm}

\begin{lema} \-\\
  Donada una variable aleatòria $X$, el nombre de discontinuïtats de $F_{X}(x)$ és un conjunt numerable.
\end{lema}

\begin{col}
  $\Phi_{X}(t)$ caracteritza completament $X$.
\end{col}

\subsection{Famílies exponencials}

Ara veurem que podem tractar de manera molt general famílies de variables aleatòries usant la noció de família exponencial.

\begin{defi}
  Una família de variables aleatòries és \textbf{exponencial} amb paràmetres $\overrightarrow{\theta} = (\theta_{1},\ldots,\theta_{n})$ si la funció de probabilitat (cas discret) o la funció de densitat (cas continuu) té la forma:
  \[
    p(x\mid \overrightarrow{\theta}) = f(x,\overrightarrow{\theta}) = g(x)\cdot exp\Big(\sum_{i=1}^{n}\theta_{i}t_{i}(x)-C(\overrightarrow{\theta})\Big) \quad \text{ (on $\setb{t_{i}(x)}_{i=1}^{n}$ són funcions l.i.)}
  \]
  La família es diu \textbf{natural} si $\exists k$ tal que $t_{k}(x)=x$.
\end{defi}

\begin{example}
  \begin{enumerate}
  \item[]
      \item $X\sim Ber(p)$
      \[
        p(x\mid p) = p^{x}(1-p)^{1-x} = exp\Big(x\cdot log(p) + (1-x)\cdot log(1-p)\Big), \quad x\in\setb{0,1}
      \]
      Agafem l'exponent:
      \[
        x\cdot log(p) + (1-x)\cdot log(1-p) = x\Big(log(p) - log(1-p)\Big) = x\bigg(\underbrace{log\Big(\frac{p}{1-p}\Big)}_{\theta} \bigg) + log(1-p)
      \]
      Si fem $\theta = log\Big(\dfrac{p}{1-p}\Big) \implies e^{\theta} = \dfrac{p}{1-p} \implies e^{\theta} - pe^{\theta} = p \implies p = \dfrac{e^{\theta}}{1+e^{\theta}}$
      \[
        \implies log(1-p) = log\bigg(1-\frac{e^{\theta}}{1+e^{\theta}}\bigg) = -log(1+e^{\theta})
      \]
      Per tant, 
      \[
        p(x\mid \theta) = exp\Big(\underbrace{x}_{t_{1}(x)}\theta - \underbrace{log(1+e^{\theta})}_{C(\theta)}\Big) \qquad \bigg(\theta = log\frac{p}{1-p}\bigg)
      \]
      és família exponencial natural.
      \item $X\sim Bin(N,p)$
      \[
        p(x\mid N,p) = \binom{N}{x}\cdot p^{x}(1-p)^{N-x} \quad x \in \setb{0,1,\ldots,N}
      \]
      $N$ no pot ser un paràmetre! $\implies N$ ha de ser fix. \\\\
      En aquesta situació: \\\\
      $
        p(x\mid p) = \underbrace{\binom{N}{x}}_{g(x)} p^{x}(1-p)^{N-x} = g_{N}(x)\cdot exp\Big(x\cdot log(p) + (N-x)\cdot log(1-p)\Big) = g_{N}(x)\cdot exp\Big(x\cdot log(\frac{p}{1-p}) + N\cdot log(1-p)\Big) \underset{\theta = log\frac{p}{1-p}}{=}
        g_{N}(x)\cdot exp\big(x\cdot\theta - N\cdot \underbrace{log(1+e^{\theta})}_{C(\theta)}\big)
      $
  \end{enumerate}
\end{example}
