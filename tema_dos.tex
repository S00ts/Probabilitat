\section{Variables Aleatòries}
\subsection{Definició de variable aleatòria. Llei d'una v.a.}
Sigui $(\Omega, \mathcal{A}, \beta)$  un espai de probabilitat. Volem estudiar funcions de $\Omega$ amb imatge en $\real$.

\begin{defi}
  Una \textbf{variable aleatòria} és una funció $X\colon \Omega \to \real$ tal que per tot borelià B 
  $\in \mathcal{B}$, $\inv{X}(B) \in \mathcal{A}$. \\
  
  Per tant, una variable aleatòria és una funció mesurable entre els espais de mesura $(\Omega, \mathcal{A}, p)$ i $(\real, \mathcal{B}, \lambda)$.
\end{defi}

\begin{example}
  (1) Les funcions constants són variables aleatòries: \\
    \[
    \begin{aligned}
      X \colon \Omega &\to \real \\
      \omega &\mapsto c
    \end{aligned}
    \qquad \text{Si prenem } B \in \mathcal{B} \text{, } \inv{X}(B) =
    \begin{cases}
			\emptyset \quad \text{si c} \notin B\\
			\Omega \quad \text{si c} \in B
    \end{cases}
    \]
  \\
  
  (2) \textbf{Variables aleatòries indicadores}: 
    \[
    \hspace{-3.5cm}\text{Sigui A}  \in \mathcal{A}\text{, definim }\mathbbm{1}_{A}\colon \Omega \to \real \text{ on } \mathbbm{1}_{A}(\omega) = 
    \begin{cases}
			0 \quad \text{si } \omega \notin A\\
			1 \quad \text{si } \omega \in A
    \end{cases}\\
    \]
    \[
    \hspace{-3.5cm}\text{Aleshores, } B \in \mathcal{B}, \inv{\mathbbm{1}_{A}}(B) = 
    \begin{cases}
			\emptyset \quad \text{si } \setb{0, 1} \nsubseteq B\\
			A \quad \text{si } 1 \in B, \quad 0\notin B\\
			\overline{A} \quad \text{si } 1 \notin B, \quad 0 \in B\\
			\Omega \quad \text{si } \setb{0, 1} \nsubseteq B
    \end{cases}\\
    \]
    \\
    
    (3) Si X i Y són v.a., aleshores $X + Y$, $X\cdot Y$, $\abs{X}$, etc. són v.a. \\
    \- \hspace{0.5cm}En general, si $g\colon \real^{2} \to \real$ és una funció mesurable, aleshores g(X,Y) és una v.a.\\
\end{example}

Estem dient que $\forall B \in \mathcal{B}$, $\setb{\omega \in \Omega \colon X(\omega) \in B}$ és un succés i, per tant, podem calcular $P(\setb{\omega \in \Omega \colon X(\omega) \in B}) \equiv P(X \in B)$.\\

\begin{example}
  $P(X\leq 1) = P(\setb{\omega \in \Omega \colon X(\omega) \in (-\infty, 1)})$\\
\end{example}

Les v.a. permeten traslladar l'estructura d'espai de probabilitat de $(\Omega, \mathcal{A}, p)$ en $(\real, \mathcal{B})$, donant lloc a mesures que no provenen de la mesura de Lebesgue.\\

\newpage

\begin{defi}
  Siguin $(\Omega, \mathcal{A}, p)$ un espai de probabilitat i X una v.a. \\
  La \textbf{mesura de probabilitat induïda} per X és una mesura de probabilitat sobre $(\real, \mathcal{B})$ definida per
  \[
    \begin{aligned}
      p_{X} \colon \mathcal{B} &\to \real \\
      B &\mapsto p_{X} = P(\setb{\omega \in \Omega \colon X(\omega) \in B})
    \end{aligned}
  \]
  \\
\end{defi}

\begin{obs}
  $(\real, \mathcal{B},p_{X})$ és un espai de probabilitat.
\end{obs}

De teoria de la mesura, és equivalent veure que [$\forall B \in \mathcal{B}, \quad \inv{X}(B) \textit{ és de } \mathcal{A}$] a veure que [\textit{l'antiimatge de qualsevol interval} $\in \mathcal{A}$].\\\\
Per tant, per saber si una funció és una v.a. només cal veure si l'antiimatge dels intervals són de $\mathcal{A}$.\\

La següent definició dóna una funció en $\real$ que codifica molta informació de X:

\begin{defi}
  Donada X v.a., la \textbf{funció de distribució de probabilitat} de X és:
  \[
    \begin{aligned}
      F_{X} \colon \real &\to [0,1] \\
      x &\mapsto P(X \leq x)
    \end{aligned} 
  \]
\end{defi}

\begin{properties}
  (i) Si $x_{1} \leq x_{2} \implies F_{X}(x_{1}) \leq F_{X}(x_{2})$\\
  \[
  \hspace{-9.4cm}\text{(ii) }\lim_{x\to -\infty} F_{X}(x) = 0 \text{ ,  } \lim_{x\to+\infty} F_{X}(x) = 1
  \]
  (iii) $F_{X}(x)$ és contínua per la dreta: $\forall x, \lim_{h\to0^{+}}F_{X}(x+h) = F_{X}(x)$
\end{properties}

\begin{obs}
  \begin{itemize}
      \item []
      \item $P(X > x) = 1 - P(X\leq x) = 1 - F_{X}(x)$
      \item $P(x_{1} < X \leq x_{2}) = P(X \leq x_{2}) - P(X \leq x_{1}) = F_{X}(x_{2}) - F_{X}(x_{1})$
  \end{itemize}
\end{obs}

\begin{obs}
  Les propietats (i), (ii), (iii) de $F_{X}(x)$ són de fet suficients.\\
  Si una funció $F(x)$ satisfà (i), (ii), (iii), aleshores és funció de probabilitat d'una variable aleatòria.
\end{obs}
\newpage
\subsection{Moments d'una v.a.  Desigualtats de Markov i Chebyshev}

Siguin $(\Omega, \mathcal{A}, p)$ uns espai de probabilitat i X una v.a.\\
\begin{defi}
  L'\textbf{esperança} de X és: 
  \[
  \E[X] = \int_{\Omega}X dp = \int_{\real} x \, dp_{X}
  \]
  \\
  Més en general, si $f\colon \real \to \real$ és una funció mesurable, 
  \[
    \E[f(x)] = \int_{\Omega}f(x) dp = \int_{\real} f(x) \, dp_{X}
  \]
\end{defi}

\begin{obs}
  De teoria de la mesura, cal recordar que una funció $g$ és integrable sii $\abs{g}$ ho és
  (En general, $\E[f(x)]$ està definida sii $\E[\abs{f(x)}]<+\infty$). \\
\end{obs}

Si particularitzem $f$:

\begin{defi}
  $f(x) = X^{r} \implies \E[X^{r}]$ és el \textbf{moment r-èssim}.
\end{defi}

\begin{defi}
  Si $\E[X] = p < +\infty$, $\E[(X-p)^{r}]$ és el \textbf{moment normalitzat r-èssim}.\\\\
  En particular, si r = 2, $\E[(X-p)^{2}] = \mathbbm{V}ar[X]$ és la \textbf{variància} de $X$.
\end{defi}

\begin{defi}
  Si $f(x) = x(x-1)\ldots(x-r+1) \implies \E[f(x)] = \E[(X)_r]$ és el \textbf{moment factorial r-èssim}.
\end{defi}

\begin{prop}[(Propietats de l'esperança i la variància)]
  \begin{itemize}
      \item []
      \item Si $c$ és la v.a. constant, $\E[c] = c$ i $\mathbbm{V}ar[c] = 0$
      \item \underline{Linealitat}: si $a, b \in \real$ i $X, Y$ v.a., $\E[aX + bY] = a\E[X] + b\E[Y]$
      \item $A \in \mathcal{A}, X = \mathbbm{1}_{A}$, $\E[\mathbbm{1}_{A}] = P(A)$
      \item $\abs{\E[X]} \leq \E[\abs{X}]$
      \item $\mathbbm{V}ar[c\cdot X] = c^{2}\cdot \mathbbm{V}ar[X]$
      \item $\mathbbm{V}ar[c+X] = \mathbbm{V}ar[X]$
      \item $\mathbbm{V}ar[X] = \E[X^{2}] - (\E[X])^{2}$
  \end{itemize}
\end{prop}

\newpage

\begin{obs}
  Si $\E[\abs{X}^{p}] < +\infty$, aleshores podem utilitzar tots els resultats de teoria dels espais $L_{p}$. Així doncs tenim les següents conseqüències:
  
  \begin{itemize}
      \item \underline{Hölder}: $p, q > 0$, $\frac{1}{p} + \frac{1}{q} = 1, \, \E[\abs{X}^{p}] < +\infty, \E[\abs{Y}^{q}] < +\infty \\ \- \hspace{1.25cm}\implies \E[\abs{XY}] \leq \E[\abs{X}^{p}]^{\frac{1}{p}}\cdot \E[\abs{Y}^{q}]^{\frac{1}{q}} \quad (\E[\abs{XY}]^{pq} \leq \E[\abs{X}^{p}]^{q}\cdot \E[\abs{Y}^{q}]^{p})$
      
      \item \underline{Cauchy-Schwartz}: si $\E[X^{2}], \E[Y^2] < +\infty$, aleshores $\E[XY]^{2} \leq \E[X^{2}]\cdot \E[Y^{2}]$
      \item \underline{Minkowski}: si $\E[\abs{X}^{p}], \E[\abs{Y}^{p}] < +\infty \implies \E[\abs{X+Y}^{p}]^{\frac{1}{p}} \leq \E[\abs{X}^{p}]^{\frac{1}{p}} + \E[\abs{Y}^{p}]^{\frac{1}{p}}$
  \end{itemize}
\end{obs}

\begin{thm}[(Desigualtat de Markov)]
  Sigui $X$ un v.a. que pren valors positius i $a > 0$. Aleshores: \\
  \[
    P(X \geq a) \leq \frac{\E[X]}{a}
  \]
\end{thm}

El següent resultat dóna estimacions quantitatives de quant es dispersa una v.a. en relació a la seva esperança:

\begin{thm}[(Desigualtat de Chebyshev)]
  Sigui X una v.a. en $(\Omega, \mathcal{A}, p)$ amb $\E[X], \V ar[X] < +\infty$. Aleshores, $\forall k > 0$
  \[
    P(\abs{X-\E[X]}\geq k\cdot \V ar[X]^{\frac{1}{2}}) \leq \frac{1}{k^{2}}
  \]
  També es pot escriure:
  \[
    P(\abs{X-\E[X]}\geq k) \leq \frac{\V ar[X]}{k^{2}}
  \]
\end{thm}

%\newpage
\subsection{Vectors de variables aleatòries. Independència de v.a.}

Donat un espai de probabilitat $(\Omega, \mathcal{A}, p)$ considerem les v.a. $X_{1}, \ldots, X_{n}$.  Cadascuna d'elles defineix una distribució de probabilitat sobre $\real$. \\
Aleshores podem considerar el vector $(X_{1}, \ldots, X_{n})\colon \Omega \to \real^{n}$. 

\begin{defi}
  Un vector $(X_{1}, \ldots, X_{n})\colon \Omega \to \real^{n}$ és un \textbf{vector de variables aleatòries} (o una v.a. multidimensional), si per tot $B \in \mathcal{B}_{n}$ (Borelians en $\real^{n}$), $\inv{(X_{1}, \ldots, X_{n})}(B) \in \mathcal{A}$. \\
  
  Com $\Pi_{i}\colon \real^{n} \to \real$ (projecció en la i-èssima component) és una funció mesurable, aleshores 
  \[
  \begin{aligned}
      \Pi_{i}(X_{1}, \ldots, X_{n}) \colon \Omega &\xrightarrow{(X_{1}, \ldots, X_{n})} \real^{n} \hspace{-0.1cm} \xrightarrow{\qquad \Pi_{i} \qquad} \real \\
      \omega &\longmapsto (X_{1}, \ldots, X_{n})(\omega) \longmapsto X_{i}(\omega)
  \end{aligned}
  \]
  $\equiv X_{i}(\omega)$ és una v.a. (en el sentit unidimensional).
\end{defi}

De la mateixa manera que vam fer per les v.a. unidimensionals, podem considerar les antiimatges només en intervals.

\begin{defi}
  Donat un espai de probabilitat $(\Omega, \mathcal{A}, p)$, i un vector de v.a. $(X_{1}, \ldots, X_{n}) = \vec{X}$, aleshores la \textbf{funció de distribució de probabilitat} de $\vec{X}$ és $F_{\vec{X}}(x_{1}, \ldots, x_{n})$ definida per:
  \[
    \begin{aligned}
      F_{\vec{X}}\colon \real^{n} &\to [0,1]\subseteq\real \\
      (x_{1}, \ldots, x_{n}) &\mapsto P\Big((X_{1}\leq x_{1}) \cap (X_{2} \leq x_{2}) \cap \ldots \cap (X_{n} \leq x_{n})\Big) = P\bigg(\bigcap\limits_{i=1}^{n} X_{i} \leq x_{i}\bigg)
  \end{aligned}
  \]
\end{defi}

Vegem propietats de la funció de distribució pel cas n = 2 (Per $ n > 2$, és idèntic):

\begin{lema} \- \\
  (i) Si $x'_{1} \geq x_{1},\, x'_{2} \geq x_{2} \implies F_{\vec{X}}(x'_{1}, x'_{2}) \geq F_{\vec{X}}(x_{1}, x_{2})$ \\
  
  $$\hspace{-4.4cm}\text{(ii)} \lim_{(x_{1}, x_{2}) \to (+\infty, +\infty)} F_{\vec{X}}(x_{1}, x_{2}) = 1  \qquad \lim_{(x_{1}, x_{2}) \to (-\infty, -\infty)} F_{\vec{X}}(x_{1}, x_{2}) = 0$$ \\
  
  $$\hspace{-3.1cm}\text{(iii)} \lim_{(h_{1}, h_{2}) \to (0^{+}, 0^{+})} F_{\vec{X}}(x_{1}+h_{1}, x_{2}+h_{2}) = F_{\vec{X}}(x_{1}, x_{2}) \quad \text{(contínua "per dalt")}$$
\end{lema}

\begin{obs}
  Aquestes 3 condicions són necessàries i suficients per a definir una v.a. multidimensional.
\end{obs}

\begin{obs}
  \begin{itemize}
      \item []
      \item Si tenim $F_{\vec{X}}(x_{1}, x_{2})$ associada a $\vec{X} = (X_{1}, X_{2})$, aleshores
      \[
        \lim_{x_{2} \to +\infty} F_{\vec{X}}(x_{1}, x_{2}) = \lim_{x_{2}\to +\infty}P\Big((X_{1}\leq x_{1})\cap (X_{2} \leq x_{2}) \Big) = P(X_{1} \leq x_{1}) = F_{\vec{X}}(x_{1})
      \]
      A aquesta funció $\Big(\lim_{x_{1} \to +\infty} F_{\vec{X}}(x_{1}, x_{2}) \Big)$ se l'anomena \textbf{funció de distribució marginal}.
      \item Prenem un rectangle en $\real^{2}$ i $\vec{X} = (X, Y)$ v.a. multidimensional:
      \[
        P(a<X\leq b, \, c < Y \leq d) = F_{\vec{X}}(b, d) - F_{\vec{X}}(a, d) - F_{\vec{X}}(b, c) + F_{\vec{X}}(a, c)
      \]
  \end{itemize}
\end{obs}

\begin{defi}
  Sigui $(\Omega, \mathcal{A}, p)$ un espai de probabilitat i $\setb{x_{i}}_{i\in I}$ un conjunt de v.a. Direm que $\setb{x_{i}}_{i\in I}$ són \textbf{independents} si: $\forall k, \, \forall {i_{1}, \ldots, i_{k}} \subseteq I, \, \forall B_{1}, \ldots, B_{k} \in \mathcal{B}$
  \[
    P(X_{i_{1}} \in B_{1}, X_{i_{2}}\in B_{2}, \ldots , X_{i_{k}}\in B_{k}) = \prod\limits_{j=1}^{k}P(X_{i_{j}}\in B_{j})
  \]
  
  Si ara prenem $X_{1}, \ldots, X_{k}$ v.a., aleshores si són independents, 
  \[
    F_{X_{1}, \ldots, X_{k}}(x_{1}, \dots, x_{k}) = P(X_{1} \leq x_{1}, X_{2} \leq x_{2}, \ldots , X_{k} \leq x_{k}) = \prod\limits_{j=1}^{k}P(X_{j} \leq x_{j}) = \prod\limits_{j=1}^{k}F_{X_{j}}(x_{j})
  \]
\end{defi}

\newpage

\begin{obs}
  Si $X$ i $Y$ són v.a. independents i $f,\, g$ funcions mesurables, aleshores $f(X)$ i $g(Y)$ són també independents.
\end{obs}

\begin{obs}
  Si $F_{X_{1}, \ldots, X_{k}}(x_{1}, \dots, x_{k}) = \prod\limits_{j=1}^{k}F_{X_{j}}(x_{j})$, aleshores les v.a. $X_{1}, \ldots , X_{k}$ són independents.
\end{obs}

En quant al càlcul de moments, tenim el següent resultat:

\begin{thm}
  Siguin $X_{1}, \ldots, x_{k}$ v.a. independents. Aleshores, si $\E[X_{i}] < +\infty$, es compleix que $$\E\bigg[\prod\limits_{i=1}^{k}X_{i} \bigg] = \prod\limits_{i=1}^{k}\E[X_{i}]$$
\end{thm}

Vam veure que l'operador esperança és lineal, però això no és cert en general per la variància. De fet, en el segon cas obtenim un terme connector anomenat covariància.

\begin{defi}
  Donades dues v.a. $X,Y$, la \textbf{covariància} de $X$ i $Y$ ($Cov(X,Y)$) és:
  \[
    Cov(X,Y) = \E\Big[(X-\E[X])\cdot (Y-\E[Y]) \Big]
  \]
  Si ara desenvolupem aquesta expressió, obtenim: $Cov(X,Y) = \E[XY] - \E[X]\E[Y]$. \\
  
  Observem que si $X$ i $Y$ són independents, aleshores $X - \E[X]$ i $Y - \E[Y]$ també ho són i, per tant:
  \[
    Cov(X,Y) = \E\Big[(X-\E[X])\cdot (Y-\E[Y]) \Big] = \E\Big[X-\E[X]\Big]\cdot \Big[Y-\E[Y] \Big] = (\E[X] - \E[X]) \cdot (\E[Y] - \E[Y]) = 0
  \]
\end{defi}

\begin{obs}
  $Cov(X,Y) = 0 \hspace{0.2cm}\not\hspace{-0.25cm} \implies X$ i $Y$ independents. %vaya chapuza
\end{obs}

Si ara calculem la variància d'una suma, obtenim: $$\V ar[X+Y] = \V ar[X] + \V ar[Y] + 2\cdot Cov(X,Y)$$ \\

\begin{obs}
  Si X i Y són independents, $Cov(X,Y) = 0$ i per tant, $\V ar[X+Y] = \V ar[X] + \V ar[Y]$. \\
  En general, això és cert per $n$ v.a. independents: 
  \[
    \V ar\bigg[\sum_{i=1}^{n}X_{i} \bigg] = \sum_{i=1}^{n}\V ar[X_{i}] \qquad \text{si } \setb{X_{i}}_{i=1}^{n} \text{ són independents}.
  \]
\end{obs}

\newpage
Vegem ara propietats de la covariància:

\begin{properties}
  (i) $Cov(c, X) = 0$ (c és una constant) \\\\
  (ii) Si $c$ és una constant, $Cov(c + X,Y) = Cov(X,Y)$ \\\\
  (iii) $Cov(X,X) = \V ar[X]$ \\\\
  (iv) $Cov(X,Y) = Cov(Y,X)$ \\\\
  (v) $Cov(aX+bY, Z) = a\cdot Cov(X,Z) + b\cdot Cov(Y,Z)$
\end{properties}

\begin{defi}
  Anomenem \textbf{coeficient de correlació de Pearson} $\big(\rho(X,Y)\big)$ a:
  \[
    \rho(X,Y) = \frac{Cov(X,Y)}{\V ar[X]^{\frac{1}{2}}\cdot \V ar[Y]^{\frac{1}{2}}} \, \in [-1,1]
  \]
\end{defi}

\begin{obs} \- \\
  Sabem que la igualtat en \textit{Cauchy-Schwartz} es dóna quan $\exists a$ tal que $a\cdot X = Y$. \\
  A més, $Cov(aX + b, Y) = Cov(aX, Y)$. 
  Per tant, $$Y=aX + b \iff \rho(X,Y) \in \setb{\pm 1}$$ Així doncs, com més proper sigui $\rho(X,Y)$ a $\pm 1$, millor serà una aproximació lineal de $Y$ usant $X$.
\end{obs}
