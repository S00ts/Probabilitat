\documentclass[12pt]{article}
 
\usepackage[margin=1in]{geometry}
\usepackage[pdftex]{hyperref}
\usepackage{amsmath,amsthm,amssymb,graphicx,mathtools,tikz,hyperref,enumerate}
\usepackage{mdframed,cleveref}
\usepackage{bbm}

\newmdenv[leftline=false,topline=false]{topright}
\let\proof\relax
\usepackage[utf8]{inputenc}
\usetikzlibrary{positioning}
\newcommand{\n}{\mathbb{N}}
\newcommand{\z}{\mathbb{Z}}
\newcommand{\q}{\mathbb{Q}}
\newcommand{\cx}{\mathbb{C}}
\newcommand{\real}{\mathbb{R}}
\newcommand{\E}{\mathbb{E}}
\newcommand{\bb}[1]{\mathbb{#1}}
\let\k\relax
\newcommand{\k}{\mathbf{k}}
\newcommand{\ita}[1]{\textit{#1}}
\newcommand\inv[1]{#1^{-1}}
\newcommand\setb[1]{\left\{#1\right\}}
\newcommand{\vbrack}[1]{\langle #1\rangle}
\newcommand{\determinant}[1]{\begin{vmatrix}#1\end{vmatrix}}
\newcommand{\abs}[1]{\left\vert #1 \right\vert}
\DeclareMathOperator{\Id}{Id}


\hypersetup{
	colorlinks,
	linkcolor=blue
}
 
 \renewcommand*\contentsname{Contenidos}

\newtheoremstyle{break}% name
{}%         Space above, empty = `usual value'
{}%         Space below
{}% Body font
{}%         Indent amount (empty = no indent, \parindent = para indent)
{\bfseries}% Thm head font
{}%        Punctuation after thm head
{\newline}% Space after thm head: \newline = linebreak
{#1 #2 \normalfont #3}%         Thm head spec

\newtheoremstyle{breakthm}% name
{}%         Space above, empty = `usual value'
{}%         Space below
{}% Body font
{}%         Indent amount (empty = no indent, \parindent = para indent)
{\bfseries}% Thm head font
{}%        Punctuation after thm head
{\newline}% Space after thm head: \newline = linebreak
{#1 \normalfont #3 (#2)\addcontentsline{toc}{subsubsection}{#1 #3}}%         Thm head spec
\newtheoremstyle{normal}% name
{}%         Space above, empty = `usual value'
{}%         Space below
{}% Body font
{}%         Indent amount (empty = no indent, \parindent = para indent)
{\bfseries}% Thm head font
{}%        Punctuation after thm head
{5pt plus 1pt minus 1pt}% Space after thm head: \newline = linebreak
{#1 #2 \normalfont #3}%         Thm head spec

\theoremstyle{normal}
\newtheorem{lema}{Lema}[subsection]
\newtheorem{obs}[lema]{Observació}

\theoremstyle{break}
\newtheorem{prop}[lema]{Proposició}
\newtheorem*{proof}{Demostració}
\newtheorem{defi}[lema]{Definició}
\newtheorem{col}[lema]{Corol·lari}
\newtheorem{ej}[lema]{Exercici}
\newtheorem{example}[lema]{Exemple}

\theoremstyle{breakthm}
\newtheorem{thm}[lema]{Teorema}



\setcounter{section}{1}

 
\begin{document}
\date{}
\setlength{\parindent}{0pt}
%\title{}
 
%\maketitle

\section{Variables Aleatòries}
\subsection{Definició de variable aleatòria. Llei d'una v.a.}
Sigui $(\Omega, \mathcal{A}, \beta)$  un espai de probabilitat. Volem estudiar funcions de $\Omega$ amb imatge en $\real$.

\begin{defi}
  Una \textbf{variable aleatòria} és una funció $X\colon \Omega \to \real$ tal que per tot borelià B 
  $\in \mathcal{B}$, $\inv{X}(B) \in \mathcal{A}$. \\
  
  Per tant, una variable aleatòria és una funció mesurable entre els espais de mesura $(\Omega, \mathcal{A}, p)$ i $(\real, \mathcal{B}, \lambda)$.
\end{defi}

\begin{example}
  (1) Les funcions constants són variables aleatòries: \\
    \[
    \begin{aligned}
      X \colon \Omega &\to \real \\
      \omega &\mapsto c
    \end{aligned}
    \qquad \text{Si prenem } B \in \mathcal{B} \text{, } \inv{X}(B) =
    \begin{cases}
			\emptyset \quad \text{si c} \notin B\\
			\Omega \quad \text{si c} \in B
    \end{cases}
    \]
  \\
  
  (2) \textbf{Variables aleatòries indicadores}: 
    \[
    \hspace{-3.5cm}\text{Sigui A}  \in \mathcal{A}\text{, definim }\mathbbm{1}_{A}\colon \Omega \to \real \text{ on } \mathbbm{1}_{A}(\omega) = 
    \begin{cases}
			0 \quad \text{si } \omega \notin A\\
			1 \quad \text{si } \omega \in A
    \end{cases}\\
    \]
    \[
    \hspace{-3.5cm}\text{Aleshores, } B \in \mathcal{B}, \inv{\mathbbm{1}_{A}}(B) = 
    \begin{cases}
			\emptyset \quad \text{si } \setb{0, 1} \nsubseteq B\\
			A \quad \text{si } 1 \in B, \quad 0\notin B\\
			\overline{A} \quad \text{si } 1 \notin B, \quad 0 \in B\\
			\Omega \quad \text{si } \setb{0, 1} \nsubseteq B
    \end{cases}\\
    \]
    \\
    
    (3) Si X i Y són v.a., aleshores $X + Y$, $X\cdot Y$, $\abs{X}$, etc. són v.a. \\
    \- \hspace{0.5cm}En general, si $g\colon \real^{2} \to \real$ és una funció mesurable, aleshores g(X,Y) és una v.a.\\
\end{example}

Estem dient que $\forall B \in \mathcal{B}$, $\setb{\omega \in \Omega \colon X(\omega) \in B}$ és un succés i, per tant, podem calcular $P(\setb{\omega \in \Omega \colon X(\omega) \in B}) \equiv P(X \in B)$.\\

\begin{example}
  $P(X\leq 1) = P(\setb{\omega \in \Omega \colon X(\omega) \in (-\infty, 1)})$\\
\end{example}

Les v.a. permeten traslladar l'estructura d'espai de probabilitat de $(\Omega, \mathcal{A}, p)$ en $(\real, \mathcal{B})$, donant lloc a mesures que no provenen de la mesura de Lebesgue.\\

\newpage

\begin{defi}
  Siguin $(\Omega, \mathcal{A}, p)$ un espai de probabilitat i X una v.a. \\
  La \textbf{mesura de probabilitat induïda} per X és una mesura de probabilitat sobre $(\real, \mathcal{B})$ definida per
  \[
    \begin{aligned}
      p_{X} \colon \mathcal{B} &\to \real \\
      B &\mapsto p_{X} = P(\setb{\omega \in \Omega \colon X(\omega) \in B})
    \end{aligned}
  \]
  \\
\end{defi}

\begin{obs}
  $(\real, \mathcal{B},p_{X})$ és un espai de probabilitat.
\end{obs}

De teoria de la mesura, és equivalent veure que [$\forall B \in \mathcal{B}, \quad \inv{X}(B) \textit{ és de } \mathcal{A}$] a veure que [\textit{l'antiimatge de qualsevol interval} $\in \mathcal{A}$].\\\\
Per tant, per saber si una funció és una v.a. només cal veure si l'antiimatge dels intervals són de $\mathcal{A}$.\\

La següent definició dóna una funció en $\real$ que codifica molta informació de X:

\begin{defi}
  
\end{defi}

\end{document}
